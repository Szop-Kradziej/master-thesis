\linespread{1.15}                                % 1.3 do interlinii 1.5


% w�asne pakiety

%===============================================================================
% Ustawienia dokumentu

\frenchspacing

% ustawienia wymiar�w

\usepackage{geometry}
\geometry{
a4paper,
inner=30mm,
outer=20mm,
top=25mm,
bottom=25mm,
headheight = 12.5mm,
footskip = 12.5mm,
}


%\oddsidemargin 20mm							% margines nieparzystych stron
%\evensidemargin 20mm							% margines parzystych stron
%%\headheight 15pt								% wysoko�� paginy g�rnej
%\topmargin 25mm									% margines g�rny
%\bottommargin 25mm

% styl paginacji
\pagestyle{fancy}
%\renewcommand{\chaptermark}[1]{\markboth{#1}{}}
%\renewcommand{\sectionmark}[1]{\markright{\thesection\ #1}{}}

% nag��wek 
\fancyhf{}
\fancyfoot[LE,RO]{\thepage}
%\fancyhead[C]{\small\nouppercase{\rightmark}}

%\fancyhead[R]{\small\nouppercase{\leftmark}}
\renewcommand{\headrulewidth}{0pt}
%\renewcommand{\footrulewidth}{0pt}

% nag��wek w~stylu plain
\fancypagestyle{plain}
{
\fancyfoot[R]{\thepage}
%\fancyhf{}
%\renewcommand{\headrulewidth}{0pt}
%\renewcommand{\footrulewidth}{0pt}
}
\setlength{\parskip}{0.5em}


\DeclareCaptionFormat{myformat}{\fontsize{9}{11}\selectfont#1#2#3}
\captionsetup{format=myformat}
\captionsetup{justification=raggedright,singlelinecheck=false}


%\pagestyle{fancy}
%\fancyhead{}
%\fancyfoot{}
%\fancyfoot[R]{\thepage\ifodd\value{page}\else\hfill\fi}
%\fancyfoot[L]{\thepage\ifeven\value{page}\else\hfill\fi}


\usepackage{indentfirst}
\setlength{\parindent}{0.5cm}

% ta sekwencja tworzy czyste kartki na stronach po \cleardoublepage
\makeatletter
\def\cleardoublepage{\clearpage\if@twoside \ifodd\c@page\else
\hbox{}
\vspace*{\fill}
\thispagestyle{empty}
\newpage
\if@twocolumn\hbox{}\newpage\fi\fi\fi}
\makeatother

\usepackage{sectsty}
\sectionfont{\fontsize{13}{15}\selectfont}
\subsectionfont{\fontsize{12}{15}\selectfont}


\usepackage{titlesec}
\titleformat{\chapter}[display]
{\normalfont%
\huge% %change this size to your needs for the first line
\bfseries}{\chaptertitlename\ \thechapter}{14}{%
\Huge %change this size to your needs for the second line
}

%===============================================================================
% Zmienne �rodowiskowe i~polecenia

% definicja
\newtheorem{definition}{Definicja}[chapter]

% twierdzenie
\newtheorem{theorem}{Twierdzenie}[chapter]

% obcoj�zyczne nazwy
\newcommand{\foreign}[1]{\emph{#1}}

% pozioma linia
\newcommand{\horline}{\noindent\rule{\textwidth}{0.4mm}}

% wstawianie obrazk�w {plik}{caption}{opis}


%===============================================================================
% ustawienia pakietu hyperref

\hypersetup{
colorlinks,
citecolor=black,
filecolor=black,
linkcolor=black,
urlcolor=blue
}
\hypersetup
{
%colorlinks=true,			% false: boxed links; true: colored links
%linkcolor=black,			% color of internal links
%citecolor=black,			% color of links to bibliography
%filecolor=black,			% color of file links
%urlcolor=black			% color of external links
}
