\chapter{Badania i weryfikacja}
\label{chapter:research}

W ramach badania i weryfikacji platformy zostały przeprowadzone symuacje wykonania projektów oraz testy poprawności na podstawie danych historycznych.

\section{Badania}

Badania systemu mają na celu sprawdzenie wpływu zastosowania platformy na wykonanie projektu, w tym ocenę usprawnienia pracy studentów i wpływu na ich sposób uczestnictwa w projekcie.
Badania odbyły się na co dwu osobowej grupie testowej, której zadaniem było wykonanie dwóch rodzajów projektów.
Dodatkowo badania systemu pozwoliły na ocenę łatwości i intuicyjności obsługi interfejsów oraz ocenę zadowolenia użytkowników z działania platformy.

TODO: Coś więcej napisać o Agile, dodać matrix

\subsection{Symulacja zdefiniowanego projektu}

W przypadku symulacji zdefiniowanego zadania pełny cel projektu jest znany od samego początku.
Wykonywanie kolejnych etapów przybliża studentów do wyznaczonego początkowo celu.
Taki typ projektu jest bardzo często prowadzony w ramach różnych przedmiotów.
W ramach pracy został przygotowany i zdefiniowany na platformie projekt symulujący ten rodzaj zadania.

\subsection{Symulacja złożonego środowiska}

W przypadku symulacji złożonego środowiska, znany jest ogólny cel projektu, natomiast nieznany jest pełny jego zakres.
Nowe zadania definiowane są po zakończeniu poprzednich etapów.
W tym przypadku prowadzący pełni rolę Agile Product Owner’a, który doprecyzowuje treści zadań w miarę kolejnych zapytań studentów.
Taka symulacja ma za zadanie odzwierciedlić na grupie testowej rzeczywiste środowisko, z którym studenci spotkają się na co dzień w pracy zawodowej.

\section{Weryfikacja}

Weryfikacja działania platformy odbyła się na podstawie uruchomienia programów studentów z poprzednich semestrów.
Dane historyczne dotyczą projektu ”Podstawy programowania”.
Proces weryfikacji można podzielić na następujące kroki:
\begin{itemize}
    \item Zdefiniowanie nowego projektu na platformie wraz z przypadkami testowymi.
    \item Uruchomienie skryptów studentów z poprzednich lat na platformie.
    \item Porównanie otrzymanych wyników z wynikami i ocenami projektów studentów w poprzednim semestrze.
\end{itemize}
