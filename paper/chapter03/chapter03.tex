\chapter{Koncepcja rozwiązania}
\label{requirements}

W ramach pracy zaproponowano i zaimplementowano platformę wspomagającą weryfikację pracy studentów podczas projektu grupowego.
Omawiane narzędzie zostało utworzone jako MVP (ang. Minimal Value Product).
Oznacza to, że podczas implementacji ograniczono się do dodania głównych funkcjonalności, pozwalających na wykorzystanie narzędzia podczas prowadzenia przedmiotu realizującego projekt grupowy.
Pozostałe, dodatkowe funkcjonalności mogą zostać dodane w ramach rozwinięcia pracy.

Podstawowym zadaniem platformy jest zredukowanie czasochłonności procesu weryfikacji pracy studentów wynikającej z indywidualnego podejścia przy rozpatrywaniu cząstkowych rezultatów dla każdej z grup.
To założenie można osiągnąć poprzez automatyzację procesu sprawdzania wyników wykonania programów w ramach kolejnych etapów.
Automatyzacja sprowadzałaby się do utworzenia szeregu przypadków testowych przez prowadzącego w ramach każdego z etapów.
Następnie testy akceptacyjne zostałyby przeprowadzane na programach napisanych przez studentów.
Wynik testów byłby znany zarówno dla prowadzącego, jak i studentów.
Otrzymane rezultaty byłby jednoznaczne.
Jeśli wszystkie testy zakończyłyby się sukcesem, to dana grupa zaliczyłaby w pełni zadany etap.
W przypadku, gdy któryś z testów ukończyłby się błędem, aplikacja studencka nie spełniałaby wszystkich założeń projektowych.
Na podstawie znajomości ilości i rodzaju testów, które zakończyły się niepoprawnie, prowadzący miałby jasną sytuację co do postępów pracy danego zespołu dla zadanego etapu.
Te same testy służyłyby do weryfikacji cząstkowej pracy każdej z grup, dzięki czemu prowadzący nie musiałby już indywidualnie oceniać poprawności programów.

Wymuszenie usunięcia indywidualnego podejścia do grup nie jest w pełni możliwe i~wskazane.
Obecnie w ramach każdego etapu studenci udostępniają swoje sprawozdania z postępów prac prowadzącym.
Analiza i ocena dokumentów odbywa się indywidualnie dla każdej grupy.
Studenci zamieszczają sprawozdania korzystając z systemu kontroli wersji.
Dokumenty jednak nie zawsze znajdują się w jasno określonym, intuicyjnym miejscu w repozytorium, a odnalezienie ich zajmuje zbędny czas.
Również w przypadku kodu programu niejednokrotnie ciężko stwierdzić, która wersja jest ostateczna dla zadanego etapu.
W celu ułatwienia oceny pracy studentów należałoby zebrać w jednym, intuicyjnym miejscu trzy elementy cząstkowej pracy studentów pozwalające na jej ocenę: program wykonywalny, link do kodu aplikacji oraz sprawozdanie.
Dzięki łatwemu dostępowi do tych elementów prowadzący może uniknąć błędów podczas oceniania etapu.

Aktualnie informacja zwrotna dotycząca wyniku uzyskanego dla zadanego podzadania jest przekazywana studentom na zakończenie etapu.
Jest to stosunkowo późny moment.
Uzyskanie informacji zwrotnej przez grupę można przyspieszyć, poprzez udostępnienie im narzędzia pozwalającego na wielokrotne sprawdzenie działania programu przy użyciu zdefiniowanych przez prowadzących testów akceptacyjnych.
Umożliwiając studentom samodzielne sprawdzenie ich aplikacji, pozwalamy im na zdobycie w szybszym czasie informacji o nieprawidłowościach w ich programie.
Dzięki temu studenci mają czas na eliminację błędów i doprecyzowanie założeń na wstępnym etapie implementacji danego etapu.
Studenci powinni mieć wgląd do pełnej definicji wszystkich przypadków testowych, aby mogli lepiej zrozumieć błędne działanie swoich aplikacji.
Wśród informacji zwrotnej dla każdego z testów powinny znajdować się:
\begin{itemize}
    \item dane wejściowe,
    \item oczekiwane dane wyjściowe oraz zwrócony wynik,
    \item status,
    \item logi aplikacji.
\end{itemize}

Zadaniem grupy studentów korzystającej z platformy jest zamieszenie na niej wyników swojej pracy dla kolejnych etapów.
Należą do nich: sprawozdanie, link do kodu oraz program.
Platforma powinna umożliwiać umieszczanie dokumentu w dowolnym formacie.
Użytkownicy powinny móc w intuicyjny sposób przejść do kodu programu powstałego dla zadanego etapu poprzez link prowadzący do odpowiedniego commita w systemie kontroli wersji.
W wersji MVP nie zakłada się kompilacji kodu studentów bezpośrednio na platformie.
Oczekuje się, że studenci sami zbudują swoje projekty i zamieszczą na platformie wykonywalne programy.
Warto zaznaczyć, że umieszczane aplikacje mogą zawierać istotne błędy powodujące przykładowo wycieki pamięci i nieskończone pętle.
Platforma powinna być odporna na takie sytuacje i zwrócić błąd wykonania danego programu, nie doprowadzając do zawieszenia i zamknięcia całego systemu.

Studenci przedstawiający manualnie swoje wyniki cząstkowe prowadzącym mogą natrafić obecnie na szereg problemów.
Podczas prezentacji mogą zaistnieć komplikacje związane między innymi z niepoprawną konfiguracją środowiska uruchomieniowego lub niewłaściwym doborem przypadków testowych.
W sytuacji, gdy zadaniem studentów jest zaprezentowanie interakcji pomiędzy stworzonymi
przez nich modułami programów, prawdopodobieństwo wystąpienia powyższych błędów
wzrasta.
Proponowane narzędzie powinno przechowywać i udostępniać wspólną konfigurację środowiska uruchomieniowego w ramach projektu i umożliwiać definiowanie procesu integracji poszczególnych modułów.
W ramach procesu integracji prowadzący powinien móc zdefiniować przypadki testowe pozwalające na ocenę interakcji programów.

Wszystkie wprowadzane przez użytkowników dane dotyczące projektów, grup, integracji oraz etapów powinny być edytowalne.
Wyjątek stanowią kod, program oraz raport dla zadanego etapu.
Ich edycja powinna być możliwa przez studentów od daty rozpoczęcia do daty zakończenia danej części projektu.
Oba terminy powinny być dostępne dla użytkowników.

Platforma powinna udostępniać studentom wyniki poszczególnych etapów ich pracy.
Dla aktualnie prowadzonego etapu powinny być to pełne, opisane wyżej, wyniki testów.
W przypadku już zakończonych etapów studenci powinni mieć wgląd do podsumowanego statusu ich pracy, mówiącego o tym ile z testów akceptacyjnych zostało przez nich zaliczone.
Jak zostało wspomniane wcześniej, studenci nie pracują systematycznie.
Aby zmotywować ich do regularnych działań, platforma powinna udostępniać zbiorczy podgląd wyników innych grup, tak aby mogli oni sklasyfikować swoje postępy na tle innych.

Wprowadzane narzędzie powinno integrować się z używanym obecnie systemem kontroli wersji.
W myśl tego założenia prowadzący powinien mieć możliwość konfiguracji na platformie takich samych grup projektowych jak grupy zdefiniowane w systemie.
Dodanie zespołów projektowych powinno być proste i nie wymagać powielania tych samych czynności w obu narzędziach.
Platforma powinna umożliwiać wgranie grup bezpośrednio z pliku w formacie JSON, wyeksportowanego wcześniej z systemu kontroli wersji.

W celu lepszego zrozumienia procesu pracy studentów podczas semestru powinny zostać zbierane i przechowywane statystyki z~wykonania kolejnych etapów.
Platforma powinna umożliwiać zapisywanie prostych danych audytowych, dotyczących każdej z~prób uruchomienia programu.
W skład gromadzonych informacji powinny wchodzić:
\begin{itemize}
    \item identyfikator użytkownika,
    \item data próby,
    \item rezultat, jaki uzyskano w wyniku działania programu.
\end{itemize}
Dane powinny być przechowywane w ramach platformy i dostępne dla prowadzącego.
Opisane wyżej statystyki posłużą do analizy procesu pracy studentów i pozwolą usprawnić proces prowadzenia projektów grupowych.

Proponowane narzędzie powinno być generyczne.
Prowadzący powinien móc zdefiniować w analogiczny sposób wiele projektów, etapów, integracji, grup oraz przypadków testowych.
Obecnie w ramach programu kształcenia kładzie się nacisk na projekty grupowe.
Proponowane narzędzie powinno być uniwersalne i umożliwiać prostą definicję różnych środowisk uruchomieniowych z dowolną ich konfiguracją.
Pliki konfiguracyjne powinny być udostępnione dla użytkowników.
Takie podejście umożliwia zastosowanie platformy na każdym etapie studiów dla projektów prowadzonych w dowolnym języku programowania.

Zarówno od strony interfejsu użytkownika, jak i prowadzącego narzędzie powinno być proste i intuicyjne.
Dostępne dla użytkowników widoki powinny wyświetlać minimum niezbędnych informacji i być stworzone w myśl dobrych praktyk UX (ang. User Experience)~\cite{ux-good-practicies}.

Dostęp do funkcjonalności platformy powinien być zależny od poziomu uprawnień użytkownika.
Prowadzący powinien mieć szersze uprawnienia niż student, w tym możliwość pełnej konfiguracji projektów i podglądu danych statystycznych.
Widok studenta powinien ograniczać się do projektów, do których został przypisany.

Platforma powinna umożliwiać równoległy dostęp dla kilkudziesięciu użytkowników.
Narzędzie powinno być skalowalne i proste w konfiguracji tak, aby umożliwiać w przyszłości prowadzenie kilku projektów jednocześnie.
Czas uruchamiania platformy oraz przeprowadzania testów akceptacyjnych na programach studentów powinien być stosunkowo szybki.

Warto podkreślić, że celem wprowadzenia platformy nie jest eliminacja spotkań z~prowadzącym.
Zastosowanie narzędzia nie dąży również do zastąpienia funkcjonalności systemów do kontroli wersji, których interfejs usprawnia przeglądanie kodu.
Podstawowym zadaniem platformy jest usprawnienie procesu weryfikacji pracy postępów pracy w ramach studenckiego grupowego projektu informatycznego, zarówno dla nauczycieli akademickich, jak i studentów.






