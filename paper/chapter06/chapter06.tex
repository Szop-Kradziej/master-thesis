\chapter{Badania i weryfikacja}
\label{chapter:research}
TODO:FIX
Weryfikacja platformy odbyła się na podstawie uruchomienia na platformie historycznych programów studentów i~analizie otrzymanych rezultatów.

W celu zbadania platformy zasymulowano dwa typy projektów.
Pierwszy z nich jest projektem analogicznym do projektu wykonywanego przez studentów w~semestrze zima 2018 w~ramach przedmiotu ”Podstawy Programowania”.
Drugi typ zadania jest autorskim pomysłem i~ma na celu symulację zmiennego i~złożonego środowiska pracy.
Oba projekty zostały przeprowadzone na grupie testowej, składającej się z absolwentów Politechniki Warszawskiej.

W ramach badania i weryfikacji platformy zostały przeprowadzone symulacje wykonania projektów oraz testy poprawności na podstawie danych historycznych.

\section{Weryfikacja}

Weryfikacja działania platformy odbyła się na podstawie uruchomienia programów studentów z poprzednich semestrów.
Dane historyczne dotyczą projektu Podstawy programowania.
Proces weryfikacji można podzielić na następujące kroki:
\begin{itemize}
    \item Przeanalizowanie zadania projektowego.
    \item Lokalne uruchomienie programów.
    \item Zdefiniowanie nowego projektu na platformie wraz z przypadkami testowymi.
    \item Uruchomienie skryptów studentów z poprzednich lat na platformie.
    \item Porównanie otrzymanych rezultatów z wynikami i ocenami projektów studentów w poprzednim semestrze.
\end{itemize}

W celu weryfikacji platformy posłużono się kodem studentów napisanym w ramach przedmiotu podstawy programowania.
Zadaniem studentów było napisanie programu umożliwiającego interaktywną oraz autonomiczną grę Hey, that’s mine fish.
Programy miały za zadanie wczytać wejściowy układ planszy z pliku, wykonać zadaną opercaję oraz zapisać zmodyfikowany układ planszy do pliku.
Format pliku jest jednoznacznie określony w treści zadania i składa się z:
\begin{itemize}
    \item Wiersz 1: m n (rozmiar planszy).
    \item Wiersze od 2 do m+1: n pól odseparowanych znakiem spacji, każde z pól zawiera 2 cyfry: liczbę ryb (0-3) oraz cyfrę reprezentującą gracza (1-9 lub 0 jeśli pole nie jest zajęte).
    \item Wiersze od m+2: 3 pól, kolejno: nazwy gracza (string), cyfra reprezentująca gracza (1-9), liczba punktów danego gracza.
\end{itemize}

Zgodnie z założeniami programy mają przyjmować następujące parametry:
\begin{itemize}
    \item phase=phase\_mark, phase\_mark może przyjąć jedną z dwóch wartości: placement lub movement.
    \item penguins=N, gdzie N oznacza liczbę pingwinów (pionków) danego gracza.
    Parametr jest używany tylko w fazie rozmieszczania.
    \item inputboardfile, nazwa pliku wejściowego z układem planszy.
    \item outputboardfile, nazwa pliku wyjściowego z układem planszy.
    \item id, w przypadku podania argumentu "id" program powinien wypisać identyfikator gracza i zakończyć działanie.
\end{itemize}

Program rozwiązujący opisane wyżej zadanie można uruchomić na platformie i sprawdzić jego działanie.
Jednak przy założonych parametrach wykonania trudno jest napisać black-box testy, które pozwolą na automatyczną weryfikację programów.
Do otrzymania korzyści z użytkowania platformy należałoby zmienić założenia co do komend i przyjmowanych parametrów, tak aby można było napisać przypadki testowe.
W tym przypadku wystarczyłoby dla fazy ruchu dodać dwa parametry wykonania: położenie pingwina, którym chcemy poruszyć oraz docelowe miejsce, w które chcemy go przesunąć.
Dla fazy rozmieszczania, należałoby również podać listę położeń w których chcemy umieścić pingwiny.

W celu zbadania platformy udostępnione zostało siedem projektów, napisanych przez studentów w semestrze 2018Z i dostępnych na GitLab.
Wstępna analiza danych pozwoliła ustalić, że spośród dostępnych grup cztery z nich ukończyły zadanie.
Dwie z pozostałych grup przerwały projekt już na początku, a jedna z grup dołączyła do innej, przez co kod z ich pracy nie jest analizowany.

Podczas lokalnego uruchamiania programów studentów napotykamy na kilka problemów.
Projekt był prowadzony przez cały semestr a kod studentów sytematycznie wrzucany na GitLab.
Studenci nie używali tagowania commitów (ciężko też od nich tego wymagać na 1 roku studiów).
Z tych powodów granice wykonania kolejnych etapów są zatarte i ciężko je odtworzyć.
Zakłada się więc, że kod znajdujący się na GitLab to ostateczne wersje projektów, które powinny spełniać wymienione wyżej założenia.

Kolejnym problemem jest określenie, wewnątrz repozytorium, która wersja kodu jest ostateczna i powinna zostać zweryfikowana.
W tym przypadku można posłużyć się datą ostatniego commita, jednak nie zawsze wydaje się to być odpowiednim rozwiązaniem.
Często zdarza się, że studenci tuż przez końcem projektu (zwłaszcza na samym początku studiów) poprawiają szybko swoje rozwiązania.
Takie działania bardzo często doprowadzają do powstawania dodatkowych błędów i powrotu do poprzedniej wersji rozwiązania.
Tak więc data ostatniego commita może nie określać jednoznacznie wersji kodu która została przedstawiona oficjalnie prowadzącemu.

Kompilacja programów sprawdza się do indywidualnego przejrzenia kodu każdego z projektów.
Spośród wszystkich czterech projektów tylko jeden miał zdefiniowany i poprawny plik Makefile.
Kolejny posiadał tylko jeden plik z właściwym kodem aplikacji (rozszerzenie .c), więc jego kompilacja była prosta.
Dwa pozostałe projekty wymagały własnoręcznej kompilacji przez zdefiniowanie pliku Makefile.

Lokalne uruchomienie i ocena sposobu działania programów wymagała również indywidualnego podejścia do każdego z zespołów.
Można założyć, że w celu weryfikacji programów każdy z nich uruchomimy w trybie interaktywnym.
Następnie dla każdego wykonamy identyczne kroki i porównany otrzymane wyniki.
Jeden z programów uruchomiony w trybie interaktywnym pozwalał na wprowadzenie ruchu gracza i przeprowadzenia założonych testów.
Inny program, posiadający jeden plik z rozszerzeniem .c uruchomił się w wersji autmatycznej rozgrywki.
Sprawia to, że przetestowanie dla założonego wcześniej schematu jest niemożliwe, ponieważ nie mamy wpływu na ruch pionków.
Dodatkowo ciężko przez to ocenić, jak zachowuje się program dla przypadków brzegowych, ponieważ nie ma możliwości ustawienia pingwinów w dowolnej lokalizacji.
W tym przypadku ocena sposobu działania programu opierała się na przeanalizowaniu algorytmu AI zaimplementowanego przez studentów i porównaniu jego działania z wynikiem symulacji.

W celu uruchomienia programów na platformie został zdefiniowany nowy projekt o nazwie penguins z jednym etapem interactive i trzema przypadkami testowymi.
Tak jak zostało wspomniane wcześniej, dla obecnych założeń projektu ciężko utworzyć black-box testy pozwalające na autmatyczną weryfikację programów.
Sprawdzenie poprawności programów studentów na platformie sprowadziło się więc do skompilowania programów lokalnie, wrzucenia ich na platformę i indywidualnego przejrzenia logów z wynikami ich wykonania.
W przypadku, gdy programy nie zalogowały błędów wykonania uznaje się, że uruchamiają się poprawnie.
Wyniki przeprowadzonych testów zostały przedstawione w tabeli TODO.

Spośród czterech testowanych programów TODO.

Jak łatwo zauważyć, taki sposób przetestowania programów nie przynosi wiele korzyści.
Równocześnie pokazuje na jak wiele problemów może napotkać prowadzący podczas weryfkacji pracy studentów, mimo korzystania z narzędzia do wersjonowania kodu.
Takie komplikacje mogą wystąpić podczas sprawdzania każdego z kolejnych etapów i nałożyć się w momencie próby przeprowadzenia integracji programów.
Jednak przy niewielkiej modyfikacji zadania, można osiągnąć dużo lepszą automatyzację procesu oceny efektów pracy zespołów.
W kolejnym rozdziale zostaną omówione zmiany założeń zadania, tak aby osiągnąć korzyści wynikające z korzystania z platformy.
Opisane również zostaną wnioski z przeprowadzenia zmodyfikowanego projektu na grupie testowej.

\section{Badania}

Badania systemu mają na celu sprawdzenie wpływu zastosowania platformy na wykonanie projektu, w tym ocenę usprawnienia pracy studentów i wpływu na ich sposób uczestnictwa w projekcie.
Badania odbyły się na co dwu osobowej grupie testowej, której zadaniem było wykonanie dwóch rodzajów projektów.
Dodatkowo badania systemu pozwoliły na ocenę łatwości i intuicyjności obsługi interfejsów oraz ocenę zadowolenia użytkowników z działania platformy.

TODO: Coś więcej napisać o Agile, dodać matrix

Opis grupy testowej:
- absolwenci
- ukończyli EiTI
- mają doświadczenie w braniu udziału w projektach grupowych na wydziale (podczas studiów)
- znają podstawy programowania (nie trzeba ich uczyć)
- mają komercyjne doświadczenie
- mogą ocenić dzięki doświadczeniu czy rozwiązanie jest przydatne podczas studiów
- mogą powiedzieć jak platforma wpływa na ich pracę w porównaniu do projektu bez platformy

Dlaczego nie studenci:
- doprowadzenie platformy do postaci MVP jest czasochłonne
- MVP ciągle mogą wystąpić jakieś nieprzewidzanie zachowania błedy
- ciężko dać studentom nie przetestowaną na mniejszej grupie platformę
- źle działająca platforma mogłaby znstudenci pierwszego roku nie iechęcić studentów
- w przypadku studentów pierwszego roku mogłaby utrudnić znacznie im pracę zamiast pomóc
- nie da się wyeliminować wszystkich błędów na początkowej fazie projektu, trzeba utrzymywać i supportować aplikacje
- trzeba byłoby przy okazji nauczyć programowania osoby z pierwszego roku
- testy dla studentów pierwszego roku trwałyby cały semestr
- testowanie na studentach i dodanie dodatkowych funkcjonalności i usprawnień - rozszerzenie pracy
- studenci pierwszego roku nie mogą powiedzieć jak platforma wpływa na ich pracę w porównaniu do projektu bez platformy

\subsection{Symulacja zdefiniowanego projektu}

W przypadku symulacji zdefiniowanego zadania pełny cel projektu jest znany od samego początku.
Wykonywanie kolejnych etapów przybliża studentów do wyznaczonego początkowo celu.
Taki typ projektu jest bardzo często prowadzony w ramach różnych przedmiotów.
W ramach pracy został przygotowany i zdefiniowany na platformie projekt symulujący ten rodzaj zadania.

\subsection{Symulacja złożonego środowiska}

W przypadku symulacji złożonego środowiska, znany jest ogólny cel projektu, natomiast nieznany jest pełny jego zakres.
Nowe zadania definiowane są po zakończeniu poprzednich etapów.
W tym przypadku prowadzący pełni rolę Agile Product Owner’a, który doprecyzowuje treści zadań w miarę kolejnych zapytań studentów.
Taka symulacja ma za zadanie odzwierciedlić na grupie testowej rzeczywiste środowisko, z którym studenci spotkają się na co dzień w pracy zawodowej.




