\chapter{Wnioski}
\label{chapter:conclusion}

TODO: Przeredagować i rozszerzyć o wnioski z badań i weryfikacji

Zastosowanie platformy w~celu weryfikacji pracy studentów niesie ze sobą wiele korzyści.
Podstawową zaletą płynącą z jej użycia jest automatyzacja procesu udostępniania oraz sprawdzania poprawności programów studentów.
Przyśpiesza ona proces sprawdzania kolejnych etapów prac i umożliwia zwiększenie ich liczby.
Student korzystający z platformy uzyskuje szybką informację zwrotną dotyczącą poprawności działania jego kodu poprzez wgląd do raportu wyników testów automatycznych.
GUI studenta przedstawia również informację o zbiorczych postępach innych grup.
Podgląd statusu prac pozostałych zespołów mógłby być dodatkową motywacją dla studentów do wykonania danego etapu.
Dzięki platformie prowadzący otrzymuje ujednolicony wgląd w bieżące postępy studentów, co za tym idzie nie ma potrzeby rozpatrywać już kolejnych grup indywidualnie.
Platforma testowa rozwiązuje też problem różnych konfiguracji systemów.
W ramach danego projektu studenci mają dostęp do skonfigurowanego wcześniej kontenera Dockerowego, na którym żądane jest, aby zadziałał ich projekt.
Samo środowisko jest wyeksportowane dla studenta, tak aby mógł testować swój program lokalnie.
Platforma umożliwia także zbieranie danych statystycznych dotyczących procesu pracy studentów.
Te informacje mogą posłużyć dalszym usprawnieniom weryfikacji pracy grup projektowych.

W ramach rozszerzenia funkcjonalności pracy dyplomowej można rozpatrzyć dodanie modułu służący do przedstawienia graficznego postępów studentów, między innymi w postaci wykresów.

Możliwym rozszerzeniem jest również dodanie moduł generatora testów.
W ramach pracy założono, że prowadzący manualnie definiuje uruchamiane przypadki testowe.
Dodatkowym rozszerzeniem pracy mogłaby być automatyzacja procesu.
Zamiast definiowania szeregu przypadków testowych w postaci plików wejściowych i oczekiwanych plików wyjściowych można byłoby do kontenera dołączyć wzorcowy program napisany przez prowadzącego.
Ten moduł miałby za zadanie dla losowych danych wejściowych o zadanej strukturze porównywać wynik działania programu studentów z wynikiem otrzymanym ze wzorcowego programu dla tych samych danych.


- Warto dodać executor do porównywania wyników porwadzącego bo na razie platforma sprawdza tylko czy dwa pliki są identyczne

- Sprawdzenie czy identyczne może nie być słuszne w trakcie gdy mamy do zaimplementowania algorytm niestabilny

- Np. zadanie implementacja quick sort w przypadku dwóch takich samych danych może dać różne wyniki (na różnych pozycjach w liście wynikowej mogą znaleźć się dwie takie same dane)