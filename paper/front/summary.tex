\newpage
\vspace{10cm}

\newpage
\begin{center}
	\textbf{Projekt i analiza platformy wspomagającej weryfikację pracy studentów
nad zespołowym projektem informatycznym}
\end{center}
\noindent{\textbf{Streszczenie}} \newline

W ramach pracy magisterskiej zdefiniowano, zaimplementowano i zbadano platformę wspomagającą weryfikację pracy studentów.
Jest ona nowym rozwiązaniem.
Powstała na podstawie analizy istniejących narzędzi komercyjnych pozwalających na weryfikację działania aplikacji.
Oparta jest również na założeniach metodyki Agile i~Extreme programming oraz na analizie trendów co do umiejętności, jakie powinien posiadać programista dla obecnego rynku IT.

Kluczowa koncepcja platformy opiera się na wykorzystaniu zestawu testów akceptacyjnych do weryfikacji pracy studentów.
Dzięki takiemu podejściu możliwe jest osiągnięcie wielu korzyści takich jak ograniczenie indywidualnego podejścia do grup, zmniejszenie czasochłonności procesu weryfikacji, uzyskanie dostępu do bieżącej informacji zwrotnej z~rezultatem uruchomienia programów.
Platforma jest dostosowana do wymagań pracy zespołowej między innymi przez udostępnienie studentom wchodzącym w skład grupy zarządzania zasobami całego zespołu.
Wśród tych zasobów można wyróżnić kod programu, raporty i programy wykonywalne.
Oczywiście narzędzie może być również użyte w~przypadku projektu indywidualnego.

Platforma używa narzędzia Docker, dzięki czemu jest generycznym rozwiązaniem wspomagającym weryfikację pracy nad projektem w dowolnym języku programowania.

W ramach pracy zbadano historyczne programy studentów i dokonano próby uruchomienia ich na platformie.
Narzędzie zostało również zbadane na testowej grupie absolwentów, która uczestniczyła w symulacji przeprowadzenia dwóch hipotetycznych projektów studenckich opracowanych na potrzeby tej pracy. \newline



\textit{\textbf{Słowa kluczowe:}} weryfikacja pracy studentów, metody kształcenia, automatyzacja projektów grupowych, prowadzenie zespołowego projektu informatycznego.

	\vspace{1cm}

\vfill
\pagebreak

\begin{center}
    \textbf{Project and analysis of a student's work verification assistance platfrom for an information technology group project}
\end{center}
\noindent{\textbf{Summary}} \newline

As part of the thesis, a platform supporting the verification of student work was defined, implemented and examined.
It is a new solution.
Platform was created on the basis of an analysis of existing commercial tools enabling verification of the application's operation.
It is also based on the assumptions of the Agile and Extreme programming methodology and on the analysis of trends as to the skills that a programmer should have for the current IT market.

The key concept of the platform is based on the use of a set of acceptance tests to verify students' work.
Thanks to this approach, it is possible to achieve many benefits such as reducing individual approach to groups, reducing the time-consuming process of verification, gaining access to current feedback with the result of running programs.
The platform is adapted to the requirements of team work, among others, by making available to students from the resource management group of the entire team.
These resources include program code, reports and executable programs.
Of course, the tool can also be used for individual projects.

The platform uses the Docker tool, making it a generic solution that supports verification of project work in any programming language.

As part of the work, historical student programs were examined and attempts were made to launch them on the platform.
The tool was also examined on a test group of graduates who participated in the simulation of carrying out two hypothetical student projects developed for the purpose of this work.\newline

\textit{\textbf{Keywords:}} student's work verification, education methods, group projects automation, group information technology project conduction.

	\vspace*{\stretch{1}}
\cleardoublepage
