\newpage
\vspace{10cm}

\newpage
\begin{center}
	\textbf{Projekt i analiza platformy wspomagającej weryfikację pracy studentów
nad zespołowym projektem informatycznym}
\end{center}
\noindent{\textbf{Streszczenie}}

W trakcie przebiegu studiów informatycznych, studenci niejednokrotnie uczestniczą w projektach grupowych.
Proces weryfikacji ich pracy, na takich zajęciachm jest czasochłonny i obecnie wymaga indywidualnego podejścia do każdego z zespołów.
Informacja zwrotna o poprawności wykonania przez studentów kolejnych etapów dociera do nich po jego ukończeniu.

Zaproponowana i zaimplementowana w ramach pracy platforma ma za zadanie zautomatyzować proces weryfikacji pracy grup poczas semestru. 
Platforma została zbadana na testowej grupie osób, które uczestniczyły w symulacji przeprowadzenia studenckiego projektu.
W ramach weryfikacji platformy uruchomiono na niej historyczne programy studentów.


\textit{\textbf{Słowa kluczowe:}} weryfikacja pracy studentów, metody kształcenia, automatyzacja projektów grupowych, prowadzenie zespołowego projektu informatycznego.

	\vspace{1cm}

\begin{center}
    \textbf{Project and analysis of student's work verification assistance platfrom for information technology group project}
\end{center}
\noindent{\textbf{Summary}}

During a computer sience studies, students attend in many group projects.
Currently veryfication of theirs work is time consuming and requires individual approach at each of the team.
Students receive a feedback about correctness of theirs work after they finish a stage.

Proposed and implemented platform helps in automation of work weryfication process.
The platform was investigated on test user group, which take part in student's project simulation.
A historicial student's programms have been used to verify the platform.

\textit{\textbf{Keywords:}} student's work verification, education methods, group projects automatization, group information technology project conduction.

	\vspace*{\stretch{1}}
\cleardoublepage
