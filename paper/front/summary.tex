\newpage
\vspace{10cm}

\newpage
\begin{center}
	\textbf{Projekt i analiza platformy wspomagającej weryfikację pracy studentów
nad zespołowym projektem informatycznym}
\end{center}
\noindent{\textbf{Streszczenie}} \newline

W ramach pracy zdefiniowano, zaimplementowano i zbadano platformę wspomagającą weryfikację pracy studentów.
Jest ona nowym rozwiązaniem.
Powstała na podstawie analizy istniejących narzędzi komercyjnych pozwalających na weryfikację działania aplikacji.
Oparta jest również o założenia metodyki Agile i Extreme programming oraz analizę trendów co do umiejętności, jakie powinien posiadać programista dla obecnego rynku IT.

Platforma wykorzystuje zestaw testów akceptacyjnych do weryfikacji pracy studentów i eliminacji indywidualnego podejścia do grup.
Używa ona narzędzia Docker, dzięki czemu jest generycznym rozwiązaniem wspomagającym weryfikację pracy nad projektem w dowolnym języku programowania.
Platforma została zaprojektowana do weryfikacji pracy nad zespołowym projektem informatycznym, jednak może zostać również użyta w przypadku projektu indywidualnego.

W ramach weryfikacji platformy uruchomiono na niej historyczne programy studentów.
Narzędzie zostało zbadane na testowej grupie absolwentów, która uczestniczyła w symulacji przeprowadzenia dwóch projektów studenckich. \newline



\textit{\textbf{Słowa kluczowe:}} weryfikacja pracy studentów, metody kształcenia, automatyzacja projektów grupowych, prowadzenie zespołowego projektu informatycznego.

	\vspace{1cm}

\vfill
\pagebreak

\begin{center}
    \textbf{Project and analysis of a student's work verification assistance platfrom for an information technology group project}
\end{center}
\noindent{\textbf{Summary}} \newline

As part of the work, the support platform was defined, implemented and tested verifying student work.
It is a new solution.
Created on the basis of knows the analysis of existing commercial tools to verify the operation application.
It is also based on the assumptions of Agile and Extreme programming methodology and analysis of trends as to the skills that a programmer should have for the current one IT market.

The platform uses a set of acceptance tests to verify student work and eliminate individual approach to groups.
It uses the Docker tool, thanks why is it a generic solution supporting the verification of work on a project in any programming language.
The platform has been designed to fictitious work on a team IT project, however, can also be used for an individual project.

As part of the platform verification, historical student programs were launched on it shapes.
The tool was tested on a test group of graduates that participated in the simulation of carrying out two student projects. \newline

\textit{\textbf{Keywords:}} student's work verification, education methods, group projects automatization, group information technology project conduction.

	\vspace*{\stretch{1}}
\cleardoublepage
