\newpage
\vspace{10cm}

\newpage
\begin{center}
	\textbf{Projekt i analiza platformy wspomagającej weryfikację pracy studentów
nad zespołowym projektem informatycznym}
\end{center}
\noindent{\textbf{Streszczenie}} \newline

W ramach pracy magisterskiej zdefiniowano, zaimplementowano i zbadano platformę wspomagającą weryfikację pracy studentów.
Jest ona nowym rozwiązaniem.
Powstała na podstawie analizy istniejących narzędzi komercyjnych pozwalających na weryfikację działania aplikacji.
Oparta jest również na założeniach metodyki Agile i Extreme programming oraz na analizie trendów co do umiejętności, jakie powinien posiadać programista dla obecnego rynku IT.

Kluczowa koncepcja platformy opiera się na wykorzystaniu zestawu testów akceptacyjnych do weryfikacji pracy studentów.
Dzięki takiemu podejściu możliwe jest osiągnięcie wielu korzyści takich jak ograniczenie indywidualnego podejścia do grup, zmniejszenie czasochłonności procesu weryfikacji, uzyskanie dostępu do bieżącej informacji zwrotnej z~rezultatem uruchomienia programów.
Platforma jest dostosowana do wymagań pracy zespołowej między innymi przez udostępnienie studentom wchodzącym w skład grupy zarządzania zasobami całego zespołu.
Wśród tych zasobów można wyróżnić kod programu, raporty i programy wykonywalne.
Oczywiście narzędzie może być również użyte w~przypadku projektu indywidualnego.

Platforma używa narzędzia Docker, dzięki czemu jest generycznym rozwiązaniem wspomagającym weryfikację pracy nad projektem w dowolnym języku programowania.

W ramach pracy zbadano historyczne programy studentów i dokonano próby uruchomienia ich na platformie.
Narzędzie zostało również zbadane na testowej grupie absolwentów, która uczestniczyła w symulacji przeprowadzenia dwóch hipotetycznych projektów studenckich opracowanych na potrzeby tej pracy. \newline



\textit{\textbf{Słowa kluczowe:}} weryfikacja pracy studentów, metody kształcenia, automatyzacja projektów grupowych, prowadzenie zespołowego projektu informatycznego.

	\vspace{1cm}

\vfill
\pagebreak

\begin{center}
    \textbf{Project and analysis of a student's work verification assistance platfrom for an information technology group project}
\end{center}
\noindent{\textbf{Summary}} \newline
TODO: Poprawić jak wersja PL będzie zatwierdzona

As part of the master thesis, the platform to support verifying student work was defined, implemented and examined.
The tool is a new solution.
It is created on the basis of knows the analysis of existing commercial tools to verify applications.
It is also based on the assumptions of Agile and Extreme programming methodology and analysis of trends of skills that a programmer should have for the current IT market.

The platform uses a set of acceptance tests to verify students work and eliminate individual approach to groups.
It uses the Docker tool, thanks to that it is a generic solution supporting the verification of work for a project written in any programming language.
The platform has been designed to verify work on a team IT project, however, can also be used for an individual project.

As part of the platform verification, historical student programs were launched on it.
The tool was examined on a test group of graduates that participated in the simulation of carrying out two student's projects. \newline

\textit{\textbf{Keywords:}} student's work verification, education methods, group projects automatization, group information technology project conduction.

	\vspace*{\stretch{1}}
\cleardoublepage
