\chapter{Cel pracy}

W~ramach pracy rozpatrywany jest problem weryfikacji aktywności i~działań studentów w ramach przedmiotu, dla którego mamy zdefiniowane kilka grup projektowych.
Zadaniem każdego z~zespołów jest wykonanie, podczas trwania semestru, tego samego projektu.
Studenci przybliżają się do jego ukończenia przez zaliczanie kolejnych etapów.
Przykładem takiego przedmiotu są miedzy innymi ”Podstawy Programowania”, prowadzone na wydziale EiTI.

Proces prowadzenia projektów grupowych jest aktualnie usprawniany.
Wśród tych usprawnień moża wyróżnić między innymi zobligowanie studentów do zamieszczania swojego kodu do systemu kontroli wersji.
Jest to istotne usprawnienie, jednak studenci (zwłaszcza pierwszego roku) nie potrafią w pełni wykorzystać zalet sytemu.
Dodatkowo usprawnienie tej postaci nie niweluje wszystkich problemów związanych z procesem oceny pracy studentów w trakcie semestru.
Obecnie proces weryfikacji cząstkowych wyników jest czasochłochonny.
Nie tylko ze względu na dużą liczbę grup, ale również z powodu indywidualnego rozpatrywania postępów poszczególnych zespołów.
Istotnym problemem jest również to, że studenci nie pracują systematycznie podczas semestru.
Może to wynikać częściowo z braku posiadania oficjalnej wiedzy o~aktualnych postępach innych grup.
Przedstawienie prowadzącym wyników cząstkowych oraz końcowych przez studentów również może okazać się czasochłonne i~kłopotliwe.
Manualne uruchamianie programów na prywatnych maszynach studentów zajmuje czas.
Jednak często podczas prezentacji prac zdarzają się błędy np. niepoprawna konfiguracja środowiska, niewłaściwie wybrany plik z przypadkiem testowym czy problem z siecią.
W przypadku gdy studenci mają za zadanie zaprezentowania integracji swoich programów prawdopodobieństwo wystąpienia błędów rośnie.
W rozdziale \ref{intro} zostały szerzej opisane motywacje do powstania pracy.

W ramach pracy zaproponowano, zaimplementowano i zbadano narzędzie umożliwiające automatyzację procesu weryfikacji pracy studentów.
Opis techniczny platformy testowej, będącej proponowanym rozwiązaniem, znajduje się w rodziale \ref{chapter:platform-technical}.
Rozdział \ref{{chapter:interfaces} zawiera opisy i instrukcje użytkowania dostępnych dla prowadzącego i studentów interfejsów.

Badania i weryfikacja platformy zostały opisane w rozdziale \ref{chapter:research}.
Weryfikacja odbyła się na podstawie historycznych programów studentów.




