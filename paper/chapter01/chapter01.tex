\chapter{Cel pracy}

Obecnie proces weryfikacji cząstkowych wyników pracy studentów nad zespołowym projektem informatycznym jest czasochłonnym zadaniem, nawet w przypadku, gdy celem każdej z~grup jest wykonanie takiego samego projektu.
Nie tylko ze względu na liczne zespoły zajęciowe, ale również z powodu potrzeby indywidualnego rozpatrywania rezultatów poszczególnych grup.

Jednocześnie z punktu widzenia studentów przedstawienie prowadzącym wyników cząstkowych oraz końcowych może okazać się kłopotliwe.
Manualne uruchamianie programów na prywatnych maszynach zajmuje czas.
Często podczas prezentacji prac występują nieoczekiwane błędy np. niepoprawna konfiguracja środowiska, niewłaściwie wybrany plik z~przypadkiem testowym czy problem z siecią.
W przypadku gdy studenci proszeni są o~zaprezentowanie integracji swoich modułów, prawdopodobieństwo wystąpienia błędów rośnie.
Innym napotykanym problemem jest fakt, że studenci nie pracują systematycznie podczas semestru.
Może to wynikać z braku doświadczenia w efektywnym planowaniu pracy, nieumiejętności szacowania czasochłonności i nieświadomości konsekwencji.

Powyższe zagadnienia są tylko częścią trudności, z~jakimi spotykają się prowadzący oraz studenci podczas uczestnictwa w zespołowym projekcie informatycznym.

Proponowanym rozwiązaniem powyżej przedstawionych problemów jest wprowadzenie platformy wspomagającej proces pracy dla grup studentów i jego weryfikację.
Stworzenie narzędzia wymagało realizacji następujących celów szczegółowych:
\begin{itemize}
    \item Analizy wymagań i specyfiki procesu nauczania zorientowanego na projekt zespołowy.
    \item Opracowania i implementacji interfejsów webowych dla użytkowników: prowadzących oraz studentów.
    \item Opracowania i implementacji serwisu umożliwiającego wykonywanie akcji przez użytkowników.
    \item Zbadania i próby uruchomienia historycznych programów studentów na platformie.
    \item Zbadania platformy dla grupy testowej poprzez symulację przeprowadzenia hipotetycznego projektu studenckiego.
    \item Analizy uzyskanych wyników i sformułowania wniosków.
\end{itemize}

Warto zaznaczyć, że celem pracy nie jest opracowanie systemu sprawiedliwej oceny członków zespołu projektowego.
Proponowane rozwiązanie może jednak stanowić ważną pomoc w procesie weryfikacji pracy studentów.

W rozdziale \ref{chapter:intro} zostały szerzej omówione motywacje do powstania pracy.
Zamieszono tam również wnioski z analizy istniejących i badanych aktualnie podejść do weryfikacji aplikacji w Polsce i na świecie.
Trzecia sekcja pracy omawia koncepcję proponowanego rozwiązania oraz założenia projektowe.
Rozdział \ref{chapter:interfaces} zawiera szczegółowe opisy i~instrukcje użytkowania dostępnych w ramach proponowanego rozwiązania interfejsów webowych (prowadzącego oraz studenta).
Opis techniczny platformy znajduje się w~rozdziale \ref{chapter:platform-technical}.
Rozdział \ref{chapter:verify} omawia przebieg oraz rezultaty otrzymane podczas badania historycznego projektu realizowanego w ramach przedmiotu ”Podstawy Programowania” na wydziale Elektroniki i Technik Informacyjnych Politechniki Warszawskiej.
Kolejny rozdział opisuje badanie platformy na grupie testowej składającej się z absolwentów tego wydziału.
W~rozdziale \ref{chapter:conclusion} zostały przedstawione wnioski z pracy dyplomowej.
Omówiono również możliwości usprawnienia platformy i~dodania nowych funkcjonalności, które mogłyby posłużyć do rozszerzenia pracy.





