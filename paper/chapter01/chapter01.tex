\chapter{Cel pracy}

Obecnie proces weryfikacji cząstkowych wyników pracy studentów nad zespołowym projektem informatycznym jest czasochłonny.
Nawet w przypadku, gdy celem każdej z~grup jest wykonanie tych samych zadań.
Czasochłonna weryfikacja wynika między innymi z indywidualnego rozpatrywania rezultatów poszczególnych zespołów.

Przedstawienie prowadzącym wyników cząstkowych oraz końcowych przez studentów również może okazać się kłopotliwe.
Manualne uruchamianie programów na prywatnych maszynach zajmuje czas.
Często podczas prezentacji prac występują nieoczekiwane błędy np. niepoprawna konfiguracja środowiska, niewłaściwie wybrany plik z~przypadkiem testowym czy problem z siecią.
W przypadku gdy studenci proszeni są o~zaprezentowanie integracji swoich modułów prawdopodobieństwo wystąpienia błędów rośnie.

Kolejnym napotykanym problemem jest fakt, że studenci nie pracują systematycznie podczas semestru.
Może to wynikać z braku doświadczenia w efektywnym planowaniu pracy, nieumiejętności szacowania czasochłonności i nieświadomości konsekwencji.

Powyższe zagadnienia są tylko częścią trudności z~jakimi spotykają się prowadzący oraz studenci podczas uczestnictwa w zespołowym projekcie informatycznym.

Proponowanym rozwiązaniem powyżej przedstawionych problemów jest wprowadzenie platformy wspomagającej.
Stworzenie narzędzie wymagało realizacji następujących celów szczegółówych:
\begin{itemize}
    \item Analizy wymagań i specyfiki procesu nauczania zorientowanego na projekt zespołowy.
    \item Opracowania i implementacji interfejsów webowych dla użytkowników: prowadzącego oraz studentów.
    \item Opracowania i implementacji serwisu umożliwiającego wykonywanie akcji przez użytkowników.
    \item Weryfikacji platformy przy pomocy danych historycznych.
    \item Zbadania platformy dla grupy testowej poprzez symulację przeprowadzenia historycznego projektu studenckiego.
    \item Analizy uzyskanych wyników i sformułowania wniosków.
\end{itemize}

Warto zaznaczyć, że celem pracy nie jest opracowanie systemu sprawiedliwej oceny członków zespołu projektowego.
Proponowane rozwiązanie może jednak stanowić ważną pomoc w procesie weryfikacji pracy studentów.

W rozdziale \ref{chapter:intro} zostały szerzej omówione motywacje do powstania pracy.
Zamieszono tam również wnioski z analizy istniejących i badanych aktualnie podejść do weryfikacji programów w Polsce i na świecie.
Trzecia sekcja pracy omawia koncepcję proponowanego rozwiązania oraz założenia projektowe.
Rozdział \ref{chapter:interfaces} zawiera szczegółowe opisy i~instrukcje użytkowania dostępnych w ramach proponowanego rozwiązania interfejsów webowych (prowadzącego oraz studenta).
Opis techniczny platformy znajduje się w~rozdziale \ref{chapter:platform-technical}.
Rozdział \ref{chapter:research} opisuje przebieg oraz rezultaty otrzymane podczas badań i weryfikacji narzędzia.
W~rozdziale \ref{chapter:conclusion} zostały przedstawione wnioski z pracy dyplomowej.
Omówiono również możliwości usprawnienia platformy i~dodania nowych funkcjonalności, które mogłyby posłużyć do rozszerzenia pracy.





