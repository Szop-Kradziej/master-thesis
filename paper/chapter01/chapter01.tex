\chapter{Cel pracy}

Obecnie proces weryfikacji cząstkowych wyników pracy studentów nad zespołowym projektem informatycznym jest czasochłonny.
Nawet w przypadku gdy celem każdej z~grup jest wykonanie, podczas trwania semestru, tch samych zadań.
Nie tylko ze względu na dużą liczbę studentów uczestniczących w zajęciach, ale również z powodu indywidualnego rozpatrywania rezultatów poszczególnych zespołów.
Przedstawienie prowadzącym wyników cząstkowych oraz końcowych przez studentów również może okazać się czasochłonne i~kłopotliwe.
Manualne uruchamianie programów na prywatnych maszynach zajmuje czas.
Często podczas prezentacji prac występują nieoczekiwane błędy np. niepoprawna konfiguracja środowiska, niewłaściwie wybrany plik z~przypadkiem testowym czy problem z siecią.
W przypadku gdy studenci proszeni są o~zaprezentowanie integracji swoich modułów prawdopodobieństwo wystąpienia błędów rośnie.
Kolejnym napotykanym problemem jest fakt, że studenci nie pracują systematycznie podczas semestru.
Może to wynikać częściowo z~braku posiadania przez nich oficjalnej wiedzy o~aktualnych postępach innych grup.
Powyższe zagadnienia są tylko częścią trudności z~jakimi spotyka się prowadzący podczas procesu weryfikacji pracy studentów.
W rozdziale \ref{intro} zostały szerzej omówione motywacje do powstania pracy.

Proponowanym rozwiązaniem powyżej przedstawionych problemów jest wprowadzenie platformy, wspomagającej weryfikację pracy studentów.

W ramach pracy zaproponowano, zaimplementowano i zbadano narzędzie pozwalające na lepszą automatyzację procesu oceny postępów uczniów.
Stworzona platforma składa się z trzech odrębnych modułów:
\begin{itemize}
    \item GUI (ang. Graphical User Interface) prowadzącego, będące interfejsem webowym. Umożliwia ono zarządzanie projektami oraz podgląd postępów studentów.
    \item GUI studenta, będące interfejsem webowym. Pozwala na dodawanie i~uruchamianie programów oraz przeglądanie otrzymanych wyników.
    \item Serwis umożliwiający przechowywanie danych oraz uruchamianie programów dla wskazanych przypadków testowych.
\end{itemize}
Rozdział \ref{chapter:interfaces} zawiera szczegółowe opisy i~instrukcje użytkowania dostępnych dla prowadzącego i~studentów interfejsów (GUI prowadzącego oraz GUI studenta).
Opis techniczny platformy, będącej proponowanym rozwiązaniem, znajduje się w~rozdziale \ref{chapter:platform-technical}.

Weryfikacja platformy odbyła się na podstawie uruchomienia na platformie historycznych programów studentów i~analizie otrzymanych rezultatów.

W celu zbadania platformy zasymulowano dwa typy projektów.
Pierwszy z nich jest projektem analogicznym do projektu wykonywanego przez studentów w~semestrze zima 2018 w~ramach przedmiotu ”Podstawy Programowania”.
Drugi typ zadania jest autorskim pomysłem i~ma na celu symulację zmiennego i~złożonego środowiska pracy.
Oba projekty zostały przeprowadzone na grupie testowej, składającej się z absolwentów Politechniki Warszawskiej.

Rozdział \ref{chapter:research} opisuje przebieg oraz rezultaty otrzymane podczas badań i weryfikacji narzędzia.

W~rozdziale \ref{chapter:conclusion} zostały przedstawione wnioski z~implementacji i~badań platformy.
Omówiono również możliwości usprawnienia i~dodania nowych funkcjonalności, które mogłyby posłużyć do rozszerzenia pracy.





