\chapter{Wstęp}
\label{intro}

TODO:
Rozwinięcie rodziału pierwszego
Dokładne omówienie tematu i problemów z przykładami
Abstrakcyjny opis narzędzia do wspomagania weryfikacji pracy studentów
Opis funkcjonalności z uzasadnieniem
Napisać o tym, że będzie nacisk na projekty tego typu (rozporządzenie) i dopisać coś z artykułów o metodach nauczania i o Agile.


Opis problemu:
	- usprawnienie procesu weryfikacji pracy studentów
	- te same zadania dla grup

Projekty grupowe:
	- korzyści z projektów grupowych
	- ważna umiejętność - praca w grupie
	- agile values - communication, respect
	- w myśl nowego rozporządzenia, będzie więcej projektów grupowych

Narzędzia do usprawniania procesów:
	- są powszechne
	- w formie gry i wizualizacji
	- paraller programing
	- nauka pisania testów
	- nauka programowania
	- usprawnianie wybierania ścieżek nauki
	- testy umiejętności algorytmicznych
	- są powszechne, ale skupiają się na pojedynczych, odseparowanych, indywidualnych zadaniach
	- przydałoby się coś do cało semestrowego projektu
	- coś co umożliwai ocenę całego procesu a nie tylko pojedynczych interakcji
	- narzędzie powinno być generyczne

Co chcemy osiągnąć i gdzie są trudności:
    - indywidualne podejście do studentów
    - brak feedbacku
    - czasochłonne
    - powtarzalne
    - brak informacji o statusie innych
    - chęć poznania statystyk i lepszego zrozumienia procesu

Zalety:
	- szybki feedback
	- proste sprawdzenie poprawności zrozumienia zadania i założeń
	- większa szansa na oddanie poprawnego rozwiązania
	- nauka dobrych praktyk (warto testować kod)
	- wiedza o aktualnych postępach innych
	- brak indywidualnego podejścia
	- wszelkie usprawnienia dla prowadzącego
	- zbieranie statystyk i dalsza analiza sposobu pracy studentów

Przypadki testowe:
	- testowanie jest ważną umiejętnością i pożądaną
	- testowanie potrzebne od samego początku
	- pomysły nauki TDD od razu
	- co raz większy nacisk
	- software house - "developer" a nie "programista"
	- podejście agile i ścisła współpraca między członkami zespołu (tester- programista)
	- agile values - feedback
	- umiejętności typu T
	- problem, ciężko ocenić trafność przypadków testowych studentów (indywidualne podejście)
	- problem na początku, przy braku wiedzy na temat programowania
	- nie można poświęcić całego przedmiotu (przepełniony program studiów)
	- warto "przemycać"
	- black box/white box and grey box testy
	- można się uczyć z dobrze napisanych przypadków testowych
	- czasochłonne wymyślanie przypadków testowych przez prowadzących
	- starsi studenci sami mogą pisać przypadki testowe
	- można przepuszczać programy przez wszystkie przypadki testowe studentów
	- trzeba ocenić te przypadki testowe - też czasochłonne
	- można napisać program i generować losowo dane wejściowe
	- najlepiej wykorzystać ML żeby te generowane przypadki miały duże pokrycie
	- trzeba mieć dane z poprzednich lat do uczenia
	- można to wykorzystać w celu rozszerzenia pracy




W~ramach pracy rozpatrywany jest problem weryfikacji aktywności i~działań studentów dla przedmiotu, dla którego mamy zdefiniowane kilka grup projektowych.
Zadaniem każdego z~zespołów jest wykonanie, podczas trwania semestru, tego samego projektu.
Studenci przybliżają się do jego ukończenia przez zaliczanie kolejnych etapów.
Przykładem takiego przedmiotu są miedzy innymi ”Podstawy Programowania”, prowadzone na wydziale EiTI.

\section{Trendy}

TODO:
Napisać o tym, że będzie nacisk na projekty tego typu (rozporządzenie) i dopisać coś z artykułów o metodach nauczania i o Agile.

Zastosowanie platformy w~celu weryfikacji pracy studentów niesie ze sobą wiele korzyści.
Podstawową zaletą płynącą z jej użycia jest automatyzacja procesu udostępniania oraz sprawdzania poprawności programów studentów.
Przyśpiesza ona proces sprawdzania kolejnych etapów prac i umożliwia zwiększenie ich liczby.
Student korzystający z platformy uzyskuje szybką informację zwrotną dotyczącą poprawności działania jego kodu poprzez wgląd do raportu wyników testów automatycznych.
GUI studenta przedstawia również informację o zbiorczych postępach innych grup.
Podgląd statusu prac pozostałych zespołów mógłby być dodatkową motywacją dla studentów do wykonania danego etapu.
Dzięki platformie prowadzący otrzymuje ujednolicony wgląd w bieżące postępy studentów, co za tym idzie nie ma potrzeby rozpatrywać już kolejnych grup indywidualnie.
Platforma testowa rozwiązuje też problem różnych konfiguracji systemów.
W ramach danego projektu studenci mają dostęp do skonfigurowanego wcześniej kontenera Dockerowego, na którym żądane jest, aby zadziałał ich projekt.
Samo środowisko jest wyeksportowane dla studenta, tak aby mógł testować swój program lokalnie.
Platforma umożliwia także zbieranie danych statystycznych dotyczących procesu pracy studentów.
Te informacje mogą posłużyć dalszym usprawnieniom weryfikacji pracy grup projektowych.
