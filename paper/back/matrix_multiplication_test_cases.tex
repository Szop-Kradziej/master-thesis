\textbf{Nazwa testu: \textit{Invalid input matrix}}

\begin{table}[H]
    \centering
    \caption{Definicja przypadku testowego \textit{Invalid input matrix}.}
    \scalebox{0.65}{
    \begin{tabular}{|l|l|l|}
        \hline
        \rowcolor[HTML]{EFEFEF}
        Plik wejściowy & Parametry & Oczekiwany plik wyjściowy \\ \hline
        \pbox{20cm}{
        \vspace{.5\baselineskip}
        1 2 \\
        3 0 \\
        2
        \vspace{.5\baselineskip} }            &
        \pbox{20cm}{
        \vspace{.5\baselineskip}
        1 2 \\
        3 0 \\
        2 8
        \vspace{.5\baselineskip} } &
        Invalid input matrix. \\ \hline
    \end{tabular}
    }
\end{table}


\textbf{Nazwa testu: \textit{Invalid input matrix reversed}}

\begin{table}[H]
    \centering
    \caption{Definicja przypadku testowego \textit{Invalid input matrix reversed}.}
    \scalebox{0.65}{
    \begin{tabular}{|l|l|l|}
        \hline
        \rowcolor[HTML]{EFEFEF}
        Plik wejściowy & Parametry & Oczekiwany plik wyjściowy \\ \hline
        \pbox{20cm}{
        \vspace{.5\baselineskip}
        1 2 \\
        3 0 \\
        2 8
        \vspace{.5\baselineskip} }            &
        \pbox{20cm}{
        \vspace{.5\baselineskip}
        1 2 \\
        3 0 \\
        2
        \vspace{.5\baselineskip} } &
        Invalid input matrix. \\ \hline
    \end{tabular}
    }
\end{table}

\pagebreak

\textbf{Nazwa testu: \textit{Invalid matrix sizes}}

\begin{table}[H]
    \centering
    \caption{Definicja przypadku testowego \textit{Invalid matrix sizes}.}
    \scalebox{0.65}{
    \begin{tabular}{|l|l|l|}
        \hline
        \rowcolor[HTML]{EFEFEF}
        Plik wejściowy & Parametry & Oczekiwany plik wyjściowy \\ \hline
        \pbox{20cm}{
        \vspace{.5\baselineskip}
        1 2 \\
        3 0 \\
        2 8
        \vspace{.5\baselineskip} }            &
        2 &
        Invalid matrix sizes. \\ \hline
    \end{tabular}
    }
\end{table}


\textbf{Nazwa testu: \textit{Matrix multiplication 2 2}}

\begin{table}[H]
    \centering
    \caption{Definicja przypadku testowego \textit{Matrix multiplication 2 2}}
    \scalebox{0.65}{
    \begin{tabular}{|l|l|l|}
        \hline
        \rowcolor[HTML]{EFEFEF}
        Plik wejściowy & Parametry & Oczekiwany plik wyjściowy \\ \hline
        \pbox{20cm}{
        \vspace{.5\baselineskip}
        1 2 \\
        3 0
        \vspace{.5\baselineskip} }            &
        \pbox{20cm}{
        \vspace{.5\baselineskip}
        3 0 \\
        2 2
        \vspace{.5\baselineskip} }  &
        \pbox{20cm}{
        \vspace{.5\baselineskip}
        7 4 \\
        6 0
        \vspace{.5\baselineskip}
        } \\ \hline
    \end{tabular}
    }
\end{table}


\textbf{Nazwa testu: \textit{Matrix multiplication 2 3}}

\begin{table}[H]
    \centering
    \caption{Definicja przypadku testowego \textit{Matrix multiplication 2 3}}
    \scalebox{0.65}{
    \begin{tabular}{|l|l|l|}
        \hline
        \rowcolor[HTML]{EFEFEF}
        Plik wejściowy & Parametry & Oczekiwany plik wyjściowy \\ \hline
        \pbox{20cm}{
        \vspace{.5\baselineskip}
        1 2 \\
        3 0 \\
        2 8
        \vspace{.5\baselineskip} }            &
        \pbox{20cm}{
        \vspace{.5\baselineskip}
        3 0 1 \\
        2 2 4
        \vspace{.5\baselineskip} }  &
        \pbox{20cm}{
        \vspace{.5\baselineskip}
        7 4 9 \\
        9 0 3 \\
        22 16 34
        \vspace{.5\baselineskip}
        } \\ \hline
    \end{tabular}
    }
\end{table}


\textbf{Nazwa testu: \textit{Matrix multiplication 3 2}}

\begin{table}[H]
    \centering
    \caption{Definicja przypadku testowego \textit{Matrix multiplication 3 2}}
    \scalebox{0.65}{
    \begin{tabular}{|l|l|l|}
        \hline
        \rowcolor[HTML]{EFEFEF}
        Plik wejściowy & Parametry & Oczekiwany plik wyjściowy \\ \hline
        \pbox{20cm}{
        \vspace{.5\baselineskip}
        1 2 5 \\
        3 0 3
        \vspace{.5\baselineskip} }            &
        \pbox{20cm}{
        \vspace{.5\baselineskip}
        3 0 1 \\
        2 2 4 \\
        3 2 1
        \vspace{.5\baselineskip} }  &
        \pbox{20cm}{
        \vspace{.5\baselineskip}
        22 14 14 \\
        18 6 6
        \vspace{.5\baselineskip}
        } \\ \hline
    \end{tabular}
    }
\end{table}

\pagebreak

\textbf{Nazwa testu: \textit{Matrix multiplication 3 3}}

\begin{table}[H]
    \centering
    \caption{Definicja przypadku testowego \textit{Matrix multiplication 3 3}}
    \scalebox{0.65}{
    \begin{tabular}{|l|l|l|}
        \hline
        \rowcolor[HTML]{EFEFEF}
        Plik wejściowy & Parametry & Oczekiwany plik wyjściowy \\ \hline
        \pbox{20cm}{
        \vspace{.5\baselineskip}
        1 2 5 \\
        3 0 3 \\
        2 8 1
        \vspace{.5\baselineskip} }            &
        \pbox{20cm}{
        \vspace{.5\baselineskip}
        3 \\
        2 \\
        3
        \vspace{.5\baselineskip} }  &
        \pbox{20cm}{
        \vspace{.5\baselineskip}
        22 \\
        18 \\
        25
        \vspace{.5\baselineskip}
        } \\ \hline
    \end{tabular}
    }
\end{table}

