%*******************************************************************************
% Bibliografia - spis literatury wykorzystanej przy tworzeniu pracy
%*******************************************************************************

\begin{thebibliography}{99}
\addcontentsline{toc}{chapter}{Bibliografia}
\bibliographystyle{tran}

%[1] F. Ahmed, L. F. Capretz, S. Bouktif, P. Campbell: Soft Skills and Software Development: A Reflection from Software Industry, International Journal of Information Processing and Management, May 2013.
\bibitem{soft-skills} F. Ahmed, L. F. Capretz, S. Bouktif, P. Campbell \emph{,,Soft Skills and Software Development: A Reflection from Software Industry''}, International Journal of Information Processing and Management, May 2013.

%Agile
%[2] https://agilemanifesto.org/
\bibitem{agile-manifesto} \emph{,,Agile Manifesto''}, [Online] https://agilemanifesto.org/, 25 lipca 2019.

%[3] Ustawa Prawo o Szkolnictwie Wyższym
\bibitem{higher-education-law} \emph{,,Ustawa Prawo o Szkolnictwie Wyższym''}, [Online] https://www.uw.edu.pl/wp-content/uploads/2018/09/ustawa.pdf, 25 lipca 2019.

\bibitem{mit-groups} \emph{,, MIT Admissions - Understanding the process''}, [Online] https://mitadmissions.org/apply/process/what-we-look-for/, 02 września 2019.

% Gamifikacja
\bibitem{gamification} D. Pranantha, C. Luo, F. Bellotti, A. de Gloria,  \emph{,,Designing contents for a serious game for learning computer programming with different target users''}, University of Genoa, Italy, 2011.

%[15] Codility
\bibitem{codility} \emph{,,Codility''}, [Online] https://www.codility.com/, 25 lipca 2019.

%[16] HackerRank
\bibitem{hacker-rank} \emph{,,HackerRank''}, [Online] https://www.hackerrank.com/, 25 lipca 2019.

%[17] CodinGame
\bibitem{game-coder} \emph{,,CodinGame''}, [Online] https://www.codingame.com/home, 25 lipca 2019.

%[18] Pharaller platform
\bibitem{pharaller-platform} E. Buzek, M. Kruliš \emph{,,An Entertaining Approach to Parallel Programming Education''}, Emanuel Buzek, Martin Kruliš, Charles University Prague, Czech Republic, 2018.

%[14] Teach testing thesis
\bibitem{teach-testing-thesis} L. Tang \emph{,,Data-Driven Tools for Introductory Computer Science Education''}, Rice University Huston, United States, August 2018.

% https://pl.euro-linux.com/blog/automatyzacja-z-lotu-ptaka/
\bibitem{tests-levels} \emph{,,Automatyzacja z lotu ptaka''}, [Online] https://pl.euro-linux.com/blog/automatyzacja-z-lotu-ptaka/, 25 lipca 2019.

%[5] https://insights.stackoverflow.com/survey/2019
\bibitem{stack-overflow-survey} \emph{,,Stack Overflow survery 2019''}, [Online] https://insights.stackoverflow.com/survey/2019, 24 lipca 2019.

%[6] Cracking the coding interview
\bibitem{cracking-the-coding-interview} G. Laakmann McDowell \emph{,,Cracking the coding interview''}, CareerCup, Palo Alto, United States, 10 February 2016.

%[8] Extreem Programming
\bibitem{extreem-programming} K. Beck \emph{,,Extreem Programming Explained''}, Addison-Wesley, Boston, United States, February 2010.

%[10] tests from scratch
\bibitem{tests-important} L. Passos Scatalon, J. C. Carver, R. E. Garcia, E. F. Barbosa \emph{,,Software Testing in Introductory Programming Courses''}, SIGCSE '19, February 27–March 2, 2019, Minneapolis, MN, USA, 2019.

%[9] Tests at the begging
\bibitem{test-from-scratch} E. F. Barbosa, M. A. Silva, C. K. Corte, and J. C. Maldonado \emph{,,Integrated teaching of programming foundations and software testing''}, Frontiers in Education Conference, 2008. FIE 2008. 38th Annual, pp. S1H–5, IEEE, 2008.

%[11] TDD on sstart
\bibitem{tdd-on-start} B. H. Pachulski Camara and M. A. Graciotto Silva \emph{,,A strategy to combine test-driven development and test criteria to improve learning of programming skills''}, Proceedings of the 47th ACM Technical Symposium on Computing Science Education, pp. 443–448, ACM, 2016.

%[4] Testy na początku - niemądre
\bibitem{tests-and-begginers} S. H. Edwards  \emph{,,Using software testing to move students from trial-and-error to reflection-in-action''}, ACM SIGCSE Bulletin, vol. 36, no. 1, pp. 26–30, 2004.

%[12] Overflow studies program
\bibitem{overflow-studies-program} S. Elbaum, S. Person, J. Dokulil, and M. Jorde \emph{,,Bug hunt: Making early software testing lessons engaging and aordable''}, Proceedings of the 29th international conference on Software Engineering, pp. 688–697, IEEE Computer Society, 2007.

%[13] Write tests by students=
\bibitem{write-tests-by-students} M. H. Goldwasser \emph{,,A gimmick to integrate software testing throughout the curriculum''}, ACM SIGCSE Bulletin, vol. 34, pp. 271–275, ACM, 2002.

%[20] How to create valid dockerfile
\bibitem{docker-config}  \emph{,,Dockerfile reference''}, [Online] https://docs.docker.com/engine/reference/builder, 14 sierpnia 2019.

%[21] GitHub start page
\bibitem{gitHub} \emph{,,GitHub''}, [Online] https://github.com/, 14 sierpnia 2019.

%[22] GitHub oauth app
\bibitem{gitHub-auth-basic} \emph{,,GitHub Developer - Basics of Authentication''}, [Online] https://developer.github.com/v3/guides/basics-of-authentication/, 14 sierpnia 2019.

%[23] GitHub oauth app
\bibitem{gitHub-oauth-app} \emph{,,GitHub Developer - Building OAuth Apps''}, [Online] https://developer.github.com/apps/building-oauth-apps/, 14 sierpnia 2019.

%[21] Trello start page
\bibitem{trello} \emph{,,Trello''}, [Online] https://trello.com/, 10 września 2019.

%[21] Trello start page
\bibitem{sorting} \emph{,,Sorting algorithm''}, [Online] https://en.wikipedia.org/wiki/Sorting\_algorithm, 10 września 2019.

\end{thebibliography}
\clearpage




%===============================================================================
