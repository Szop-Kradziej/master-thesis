\textbf{Nazwa testu: \textit{Invalid input matrix}}

\begin{table}[H]
    \centering
    \caption{Definicja przypadku testowego \textit{Invalid input matrix}.}
    \scalebox{0.65}{
    \begin{tabular}{|l|l|l|}
        \hline
        \rowcolor[HTML]{EFEFEF}
        Plik wejściowy & Parametry & Oczekiwany plik wyjściowy \\ \hline
        \pbox{20cm}{
        \vspace{.5\baselineskip}
        1 2 \\
        3 0 \\
        2
        \vspace{.5\baselineskip} }            &
        Brak &
        Invalid input matrix. \\ \hline
    \end{tabular}
    }
\end{table}


\textbf{Nazwa testu: \textit{Matrix transposition 2 2}}

\begin{table}[H]
    \centering
    \caption{Definicja przypadku testowego \textit{Matrix transposition 2 2}.}
    \scalebox{0.65}{
    \begin{tabular}{|l|l|l|}
        \hline
        \rowcolor[HTML]{EFEFEF}
        Plik wejściowy & Parametry & Oczekiwany plik wyjściowy \\ \hline
        \pbox{20cm}{
        \vspace{.5\baselineskip}
        1 2 \\
        3 0
        \vspace{.5\baselineskip} }            &
        Brak &
        \pbox{20cm}{
        \vspace{.5\baselineskip}
        1 3 \\
        2 0
        \vspace{.5\baselineskip} } \\ \hline
    \end{tabular}
    }
\end{table}


\textbf{Nazwa testu: \textit{Matrix transposition 2 3}}

\begin{table}[H]
    \centering
    \caption{Definicja przypadku testowego \textit{Matrix transposition 2 3}.}
    \scalebox{0.65}{
    \begin{tabular}{|l|l|l|}
        \hline
        \rowcolor[HTML]{EFEFEF}
        Plik wejściowy & Parametry & Oczekiwany plik wyjściowy \\ \hline
        \pbox{20cm}{
        \vspace{.5\baselineskip}
        1 2 \\
        3 0 \\
        2 8
        \vspace{.5\baselineskip} }            &
        Brak &
        \pbox{20cm}{
        \vspace{.5\baselineskip}
        1 3 2 \\
        2 0 8
        \vspace{.5\baselineskip} } \\ \hline
    \end{tabular}
    }
\end{table}


\pagebreak

\textbf{Nazwa testu: \textit{Matrix transposition 3 2}}

\begin{table}[H]
    \centering
    \caption{Definicja przypadku testowego \textit{Matrix transposition 3 2}.}
    \scalebox{0.65}{
    \begin{tabular}{|l|l|l|}
        \hline
        \rowcolor[HTML]{EFEFEF}
        Plik wejściowy & Parametry & Oczekiwany plik wyjściowy \\ \hline
        \pbox{20cm}{
        \vspace{.5\baselineskip}
        1 2 5 \\
        3 0 3
        \vspace{.5\baselineskip} }            &
        Brak &
        \pbox{20cm}{
        \vspace{.5\baselineskip}
        1 3 \\
        2 0 \\
        5 3
        \vspace{.5\baselineskip} } \\ \hline
    \end{tabular}
    }
\end{table}


\textbf{Nazwa testu: \textit{Matrix transposition 3 3}}

\begin{table}[H]
    \centering
    \caption{Definicja przypadku testowego \textit{Matrix transposition 3 3}.}
    \scalebox{0.65}{
    \begin{tabular}{|l|l|l|}
        \hline
        \rowcolor[HTML]{EFEFEF}
        Plik wejściowy & Parametry & Oczekiwany plik wyjściowy \\ \hline
        \pbox{20cm}{
        \vspace{.5\baselineskip}
        1 2 5 \\
        3 0 3 \\
        2 8 1
        \vspace{.5\baselineskip} }            &
        Brak &
        \pbox{20cm}{
        \vspace{.5\baselineskip}
        1 3 2 \\
        2 0 8 \\
        5 3 1
        \vspace{.5\baselineskip} } \\ \hline
    \end{tabular}
    }
\end{table}