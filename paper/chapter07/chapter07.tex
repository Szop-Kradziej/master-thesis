\chapter{Badania}
\label{chapter:research}

Badania systemu służą do sprawdzenia wpływu zastosowania platformy na wykonanie projektu, w tym ocenę usprawnienia pracy studentów i wpływu na ich sposób uczestnictwa w projekcie.
Dodatkowo badania systemu pozwoliły na ocenę łatwości i intuicyjności obsługi interfejsów oraz ocenę zadowolenia użytkowników z działania platformy.

Aby zbadać platformę zasymulowano dwa projekty.
Dla obu zadań grupa testowa od początku zna pełny cel każdego z projektów.
Wykonywanie kolejnych etapów przybliża uczestników do wyznaczonego początkowo celu.
Taki typ projektu jest bardzo często prowadzony w ramach różnych przedmiotów na wydziale EiTI.

Oba zadania zostały przeprowadzone na grupie testowej, składającej się z absolwentów Politechniki Warszawskiej.
Opis motywów doboru członków grupy testowej znajduje się w podrozdziale \ref{research_group}.

Pierwsza z symulacji jest przeprowadzeniem projektu analogicznyego do historycznego zadania wykonywanego przez studentów w~semestrze zima 2018 w~ramach przedmiotu ”Podstawy Programowania”.
Pełna definicja problemu, przebieg symulacji oraz wnioski zostały przedstawione w podrozdziale \ref{research_penguins}.

Drugie zadanie sprowadza się do napisania programu umożliwiającego przekształcenie obrazu.
Ma ono na celu zbadanie użyteczności tworzenia procesów integracji programów w ramach platformy.
Opis tej symulacji znajduje się w podrozdziale \ref{research_matrix}.

Kolejny podrozdział zawiera wnioski z nadania absolwentom uprawnień administratora podczas testowania platformy.
Okazało się, że bardzo chętnie tworzyli oni własne etapy i testowali swój kod przy użyciu narzędzia.
Doprowadziło to do wniosku, że platforma mogłaby zostać rozszerzona w przyszłości (nie tracąc dotychczasowych funkcjonalności) tak, aby możliwe było również przeprowadzenie z jej użyciem projektów w metodyce Agile.

Ostatni podrozdział \ref{research_summary} zawiera podsumowanie przeprowadzonych badań.


\section{Opis grupy testowej}
\label{research_group}

W skład grupy testowej, dla której zostały przeprowadzone obie badane symulacje projektów wchodziło dwóch absolwentów Politechniki Warszawskiej.
Obie osoby ukończyły wydział Elektroniki i Technik Informacyjnych.

Absolwenci brali udział w projektach grupowych realizowanych na wydziale EiTI i znają podstawy programowania.
Dzięki temu mają pełne zaplecze wiedzy potrzebnej do realizacji każdego z symulowanych projektów.
Mogą oni dzięki swojemu doświadczeniu ocenić czy zastosowanie platformy byłoby użyteczne podczas prowadzenia projektów studenckich.
Absolwenci potrafią także wskazać różnice w prowadzeniu projektu bez użycia platformy oraz z jej wykorzystaniem.

Obie osoby posiadają również doświadczenie komercyjne, dzięki temu są w stanie ocenić narzędzie od strony technicznej i zasugerować dodatkowe funkcjonalności mające na celu zwiększenie użyteczności platformy.

Drugą rozpatrywaną grupą osób, która mogłaby wydawać się właściwą do przetestowania użyteczności platformy są obecni studenci.
Taki pomysł zespołu testowy został jednak odrzucony ze względu na ciągły proces udoskonalania narzędzia podczas badań.

W testowanej wersji MVP ciągle mogą wystąpić nieprzewidziane zachowania platformy.
Wynika to z faktu, że w początkowej fazie projektu ciężko jest wyeliminować wszystkie błędy.
Do ich całkowitego usunięcia trzeba utrzymywać i wspierać aplikację na bieżąco.
Takie niewielkie problemy pojawiły się podczas badania platformy przez grupę absolwentów i były na bieżąco naprawiane.

Udostępnienie studentom platformy bez przetestowania jej na mniejszej grupie mogłoby spowodować rezultat odwrotny do zamierzonego.
Zamiast wykorzystać zalety rozwiązania takie osoby mogłyby zniechęcić się ze względu na występujące błędy.
Sam czas naprawy błędów mógłby nie być satysfakcjonujący dla studentów i pogłębiać flustrację.
W przypadku studentów pierwszego roku przekazanie narzędzia na takim etapie pracy zdecydowanie utrudniłoby im wykonanie projektu.

Warto również zaznaczyć, że studenci zwłaszcza pierwszego roku nie mają szerokiego doświadczenia w uczestnictwie w projektach grupowych.
Stąd wprowadzenie dodatkowych narzędzi mogłoby wzbudzać u nich początkową niechęć.
Mogłoby okazać się, że dopiero po zebraniu większego doświadczenia mogliby właściwie ocenić wpływ platformy na proces ich pracy.

Samo doprowadzenie platformy do użytecznej formy i zaimplementowanie wszystkich niezbędnych funkcjonalności było czasochłonne.
A w przypadku studentów należałoby udostępnić im platformę na pełen semestr, co znacznie wydłużyłoby proces pisania pracy.

Przekazanie platformy w obecnej wersji dla studentów i monitorowanie ich interakcji z platformą podczas pełnego projektu jest jedną ze wskazanych możliwości kontynuacji pracy.
Dzięki analizie przebiegu rzeczywistego projektu uczelnianego można byłoby zrozumieć lepiej proces pracy studentów i uzupełnić platformę o dodatkowe, pożyteczne funkcjonalności.


\section{Symulacja projektu historycznego}
\label{research_penguins}

Do zdefiniowania zadania dla symulacji projektu historycznego posłużono się projektem wykonywanym przez studentów w ramach przedmiotu ”Podstawy Programowania” w realizacji zima 2018.
Jest to ten sam typ zadania, który posłużył do weryfikacji platformy.

Przeprowadzenie symulacji sprowadza się do następujacych kroków:
\begin{enumerate}
    \item Analiza i przedefiniowanie treści projektu.
    \item Utworzenie projektu na platformie.
    \item Udostępnienie projektu grupie testowej.
    \item Analiza wykonanej pracy przez absolwentów i sformułowanie wniosków.
\end{enumerate}

Aby umożliwić przetestowanie aplikacji poprzez testy akceptacyjne zmodyfikowano treść historycznego zadania i zakres kolejnych etapów.
W podrozdziale \ref{penguins_project_definition} opisano definicję projektu, która posłużyła do przeprowadzenia symulacji.

Kolejna sekcja \ref{penguins_simulation} opisuje przebieg symulacji dla projektu historycznego.

W ramach ostatniego podrozdziału zamieszczono analizę wykonanej pracy przez absolwentów.
Opisano również wnioski z eksperymentu.


\subsection{Definicja projektu}
\label{penguins_project_definition}

Do definicji projektu posłużono się historycznym projektem wykonywanym przez studentów w ramach przedmiotu ”Podstawy Programowania”.
Pełna definicja historycznego zadania znajduje się w załączniku \ref{file:pengiuns_description}

Niestety historyczne założenia projektowe nie pozwalały na wykorzystanie w pełni potencjału automatyzacji wprowadzanego przez platformę.
Nie pozwalały one między innymi na przetestowanie aplikacji poprzez testy akceptacyjne zarówno w wesji iteraktywnej jak i w wersji AI.
Dokładana analiza problemów związanych z bezpośrednim użyciem historycznych założeń projektowych została opisana w poprzednim rozdziale \ref{chapter:verify}.
W ramach badania zmodyfikowano założenia i zakres kolejnych etapów, tak aby uzyskać korzyści z przeprowadzenia projektu z wykorzystaniem platformy.

W ramach symulacji projektu na platformie zostały utworzone trzy etapy.
Celem kolejnych zadań jest:
\begin{enumerate}
    \item wczytanie i walidacja planszy,
    \item ustawienie pingwina na wskazanym polu planszy,
    \item wykonanie zadanego ruchu pionkiem.
\end{enumerate}

Dla każdego z etapów zarówno plik wejściowy jak i wyjściowy posiada ten sam format jak w definicji historycznego projektu.
Zmienie uległy przyjmowane przez aplikacje parametry wykonania.


W ramach pierwszego etapu utworzono dwa przypadki testowe sprawdzające:
\begin{itemize}
    \item poprawność walidacji pliku wejściowego,
    \item czy w przypadku poprawnie zdefiniowanej planszy jest ona wypisywana do pliku wyjściowego.
\end{itemize}
Do definicji przypadków testowych dla pierwszego etapu nie użyto żadnych dodatkowych parametrów.
Wejściowe oraz oczekiwane wyjściowe pliki dla powyższych testów zostały zamieszczone w załącznikach od \ref{file:pengiuns_e1_invalid_input_input} do \ref{file:pengiuns_e1_read_parse_write_output}.


Dla drugiego etapu stworzono cztery przypadki testowe.
Jeden z przypadków ocenia czy aplikacja poprawnie ustawia pingwina na dozwolonym polu.
Pozostałe trzy testy sprawdzają zachowanie programu dla przypadków brzegowych, takich jak:
\begin{itemize}
    \item próba ustawienia pingwina na polu ze zbyt dużą ilością ryb,
    \item próba ustawienia pingwina poza planszą,
    \item próba ustawienia gracza, który nie istnieje dla danej rozgrywki.
\end{itemize}

Do definicji akcji ustawiania pionka dla drugiego etapu użyto cztery parametry:
\begin{itemize}
    \item \textit{id=i}, gdzie \textit{i} jest identyfikatorem gracza i przyjmuje wartość od 1 do 9,
    \item \textit{phase=placement}, gdzie \textit{placement} oznacza fazę rozmieszczania,
    \item \textit{placement\_x=j}, gdzie \textit{j} przyjmuje wartość od 0 do n-1 i oznacza położenie pingwina w poziomie,
    \item \textit{placement\_y=k}, gdzie \textit{k} przyjmuje wartość od 0 do m-1 i oznacza położenie pingwina w pionie.
\end{itemize}

Plik zawierające dane dla przypadków testowych stworzonych dla drugiego etapu zostały zamieszczone w załącznikach od \ref{file:pengiuns_e2_valid_placement_input} do \ref{file:pengiuns_e2_too_high_player_id_output}.


Dla trzeciego etapu stworzono cztery trzy testowe.
Jeden z testów ocenia czy program poprawnie przemieszcza pingwina na dozwolone pole.
Inne dwa przypadki testowe sprawdzają zachowanie aplikacji dla warunków brzegowych, takich jak:
\begin{itemize}
    \item próba przesunięcia pingwina na pole, na którym znajduje się inny pionek,
    \item próba przemieszczenia pingwina poza planszę.
\end{itemize}

Do definicji akcji przesunięcia pionka dla trzeciego etapu użyto pięciu parametrów:
\begin{itemize}
    \item \textit{id=i}, gdzie \textit{i} jest identyfikatorem gracza i przyjmuje wartość od 1 do 9,
    \item \textit{phase=move}, gdzie \textit{move} oznacza fazę ruchu,
    \item \textit{from\_x=j}, gdzie \textit{j} przyjmuje wartość od 0 do n-1 i oznacza początkowe położenie pingwina w poziomie,
    \item \textit{from\_y=k}, gdzie \textit{k} przyjmuje wartość od 0 do m-1 i oznacza początkowe położenie pingwina w pionie,
    \item \textit{direction=d}, gdzie \textit{d} przyjmuje jedną z wartość: \textit{up, down, left, right} i oznacza kierunek przesunięcia pionka.
\end{itemize}

Plik zawierające dane dla testów utworzonych dla powyższego etapu zostały umieszczone w załącznikach od \ref{file:pengiuns_e3_valid_move_input} do \ref{file:pengiuns_e2_move_to_empty_output}.


\subsection{Przebieg symulacji}
\label{penguins_simulation}

TODO: Napisać że zadanie okazało się proste dla absolwentów i ciężko podzielne więc wykonali je pojedynczo a nie w grupie.
W grupie wykonane została druga symulacja


\subsection{Podsumowanie}

TODO


\section{Symulacja projektu z procesem integracji}
\label{research_matrix}

Drugi typ symulacji ma na celu zbadanie użyteczności zaimplementowanego w ramach platformy procesu integracji.
Wykonanie pełnego zadania projektowego sprowadza się do napisania aplikacji pozwalającej na wykonywanie określonych operacji na macierzach.

Proces tej symulacji można podzielić na następujące kroki:
\begin{enumerate}
    \item Definicja treści projektu.
    \item Utworzenie projektu na platformie.
    \item Udostępnienie projektu grupie testowej.
    \item Analiza wykonanej pracy przez absolwentów i sformułowanie wniosków.
\end{enumerate}

W podrozdziale \ref{matrix_project_definition} opisano definicję symulowanego projektu.
Kolejna sekcja \ref{matrix_simulation} opisuje przebieg zadania wykonanego przez grupę testową.
W ramach ostatniego podrozdziału przeanalizowano i podsumowano badanie.


\subsection{Definicja projektu}
\label{matrix_project_definition}

W ramach symulacji projektu na platformie zostało utworzone pięć etapów.
Celem kolejnych zadań jest:
\begin{enumerate}
    \item przemnożenie macierzy przez skalar,
    \item transpozycja macierzy,
    \item rotacja macierzy o 90 stopni zgodnie z ruchem wskazówek zegara (rotacja w prawo o 90 stopni),
    \item dodanie macierzy,
    \item przemnożenie dwóch macierzy.
\end{enumerate}

W ramach każdego etapu utworzono cztery przypadki testowe sprawdzające poprawność działań.
Plik wejściowy opisuje macierz o rozmiarze n na m i jest takiego samego formatu dla każdego z zadań.
Zawiera on m wierszy, gdzie w każdym z nich znajduje się n liczb oddzielonych spacją reprezentujących elementy macierzy.

W przypadku pierwszego etapu parametrem wejściowym jest pojedyncza liczba reprezentująca skalar.
Dla drugiego i trzeciego zadania nie podawane są parametry wejściowe.
Program wykonujący dwa ostatnie etapy powinien przyjmować jako parametr wejściowy drugą macierz.
Format opisujący macierz zawartą w pliku z parametrami dla zadania cztery i pięć jest taki sam jak format pliku wejściowego.

Pliki zawierające dane dla wszystkich powyższych testów zostały zamieszczone w załącznikach od \ref{file:matrix_e1_input} do \ref{file:matrix_e5_output}.


\subsection{Przebieg symulacji}
\label{matrix_simulation}

TODO


\subsection{Podsumowanie}

TODO


\section{Udostępnienie grupie testowej platformy z prawami administratora}

W celu pełnego i dokładnego przetestowania zaimplementowanej platformy udostępniono ją z członkom grupy testowej z nadanymi prawami administratora.
Ten rodziaj testów odbył się dodatkowo, poza opisanymi wcześniej badaniami.
Dzięki nadaniu praw administratora studenci mogli sami tworzyć i definiować projekty.
W wyniku ich testów można było wychywycić nieprawidłowości w działaniu platfromy oraz ocenić intuicyjność i łątwość obsługi interfesjsów prowadzącego.
Pozwoliło to również na określenie czy integracja interfejsów z serwerem jest właściwa.

W testów okazało się, że absolwenci bardzo chętnie definiowali własne projekty, etapy, integracje.
Z zaangażowaniem dodawali oni przypadki testowe i tworzyli programy spełniające założenia kolejnych zadań.

Analiza aktywności grupy testowej po udostępnieniu jej narzędzia z rolą administratora doporowadziłą do wniosku, że platforma mogłaby zostać użyta nie tylko do polepszenia procesu weryfikacji pracy studentów podczas standardowego projektu informatycznego.
Narzędzie mogłoby również posłużyć monitorowaniu prac studentów dla projektu prowadzonego w metodyce Agile.
W związku z możliwością użycia platformy również dla takiego typu zadania w ramach pracy dokonano zgrubnej analizy przeprowadzenia tego typu projektu.

W podrozdziale \ref{agile_proposition} omówiono propozycję przeprowadzenia zwinnego projektu.
Kolejna sekcja opisuje podstawowe funkcjonalności jakie należałoby zaimplementować w kolejnej wersji platformy aby była ona użyteczna w prowadzeniu projektów w metodyce Agile.
Ostatni podrozdział zawiera krótkie podsumowanie.

\subsubsection{Propozycja przeprowadzenia zwinnego projektu z użyciem platformy}
\label{agile_proposition}

Propozycja projektu prowadzonego w metodyce zwinnej polega na tym, że na początku semestru studenci uzyskują luźno zdefiniowany, ogólny cel zadania.
W trakcie semestru ich zadaniem jest doprecyzowanie z prowadzącym szczegółów.
Dla takiego podejścia prowadzący traktowany jest jako PO (ang. Product Owner), z którym studenci odbywają regularne spotkania.

Studenci w ramach takiego projektu sami decydują o zadaniach w ich pracy i raportują o statusie ich wykonania.
Do tego procesu mogłaby posłużyć zmodyfikowana platforma, umożliwiająca studentom definicję własngo projektu.
Zdaniem grupy byłoby samodzielne stworzenie kolejnych etapów wraz z poszerzeniem wiedzy na temat projektu.
Dla każdego z kolejnych zadań studenci mieli by za zadanie napisać własne testy akceptacyjne.
Dzięki wykorzytaniu platformy prowadzoncy miałby możliwość podejrzenia wyników prac poszczególnych grup w dowolnym momencie.

W przypadku takiej metodyki nawet dla takiego samego początkowego celu projektu dla każdej z grup, wyniki pracy w ciągu semestru będą różne.
Z tego powodu wydaje się, że pełne indywidualne podejście i rozważenie postępów dla każdego z zespołów jest nieuniknione.
W rzeczywistości można by było ograniczyć i zautomatyzować ocenę poczynań studentów poprzez wykorzystanie platfromy.
Tak jak w przypadku tradycyjnego projektu tak i zadania prowadzonego w metodyce Agile prowadzący miałby jednolity wgląd w postępy studentóœ.
Kolejne etapy byłyby definiowane przez studentów w analogiczny sposób w ramach każdego z projektów.
Informacje opisujące kolejne zadania wyświetlane byłyby przy użyciu tego samego interfejsu niezależnie od projektu czy grupy.
Prowadzący na pierwszy rzut oka mógłby łatwo ocenić postępy prac w ramach jedno zespołu czy nawet wstępnie porównać zaanagażowanie w projekt pomiędzy dwoma zespołami.

Jak zostało wspomniane wcześniej napisane przez studentów testy akceptacyjne mogą nie dawać wysokiego pokrycia.
Ocena słuszności przypadków testowych przez prowadzącego jest czasochłonna.
Dzięki użyciu platformy prowadzący miałby wgląd do testów na wysokim poziomie.
Aby zrozumieć przypadek testowy nie musiałby bezepośrednio przeglądać kodu studentów.
Jego zadanie sprowadzałoby się do zrozumienia zawartości pliku wejściowego, oczekiwanego pliku wyjściowego i parametrów.
Ocena testów byłaby wciąż indywidaulna ale tu również dla każdego projektu interfejs byłby prezentowany identycznie.

\subsubsection{Dodatkowe funkcjonalności wymagające implementacji}

Osiągnięcie przedstawionych w poprzednim podrozdziale celów nie jest jednak możliwe przy użyciu platformy w obecnej wersji MVP i wymagałoby dokonania kilku modyfikacji w ramach narzędzia.
W celu wskazania pełnych zmodyfikowanych założeń projektowych należałoby przeanalizować w pełni problemy stojące za przeporwadzenim projektu w metodyce Agile.
W ramach pracy przeprowadzono zgrubną analizę takiego podejścia i wskazano główne modyfikacje jakich należałoby dokonać w narzędziu aby osiągnięcie powyższego celu było możliwe.

Jedną z podstawowych modyfikacji jest dodatnie dodatkowego poziomu uprawnień.
Obecnie w ramach platformy istnieją tylko dwie role: prowadzącego(administratora) oraz studenta(zwykłego użytkownika).
W modelu MVP zakłada się, że w ramach platformy może istnieć kilku prowadzących którzy mają dostęp do edycji każdego z projektów, ale pomimo takich uprawnień ich zachowanie jest odpowiedzialne i nie edytują oni nie swoich projektów.
Dla zwinnego projektu należałoby umożliwić wskazanej grupie studentów zarzżdanie zadanym, pustym projektem.
Wymagałoby to dodatkowej roli w bazie posiadającej szersze uprawnienia niż zwykły student.
Warto jednak zauważyć, że nie powinny one być równe z prawami administratora, który ma między innymi możliwość usunięcia całego projektu.
W celu odpowiedniej definicji nowej roli i zakresu jej uprawnień należałoby przeanalizować dokładniej problem.

Ważnym zagadnieniem jest również równoległy dostęp do zasobów platformy.
Dla obecnej wersji platformy jedynym zasobem dla którego wymagane jest strzeżenie równoległego dostępu jest uruchamianie programów.
Łatwo zauważyć, że w przypadku zwinnego projektu, również definiowanie etapów mogłoby wymagać lepszego zrównoleglenia dostępu do zasobów.
Dodatkową potrzebą byłoby dodanie obserwacji zmian dokonywanych na platformie w ramach projektu i aktualizacji interfejsów studentów podążających za tymi zmianami.

W przypadku zwinnego projektu zbieranie statystyk powinno być również bardziej obszerne niż dla tradycyjnego.
Dla wersji MVP statystyki zbierane są tylko w ramach uruchomień kolejnych programów.
W wersji podstawowej zakłada się, że prowadzący zna wymagania i definiuje projekt na początku semestru i udostępnia go studentom.
Może on dokonywać modyfikacji zadań w trakcie trwania semestru jednak z założenia nie powinny być one duże.
W przypadku gdy studenci sami mieliby zarządzać projektem należałoby zbierać również informacje dotyczące tworzenia zadań i przechowywać je nawet dla usuniętych etapów i integracji.
Takie informacje na pozwoliłyby na lepsze zrozumienie specyfiki pracy studentów w przypadku zwinnego projektu.
Mogłyby posłużyć również do oceny pracy semestralnej i porównania zaangażowania w obrębie różnych zespołów.

Inną wartą rozważenia funkcjonalnością jest dodanie komentarza dla tworzonych przez studentów przypadków testowych.
Pomimo takiego samego przedstawienia testów w ramach wszystkich projektów ich ocena byłaby dalej indywidualna.
Mogłaby być przez to czasochłonnna i kłopotliwa.
Dodanie dodatkowego komentarza dla każdego z testów mogłoby być przydatne w przypadku bardziej złożonych przypadków testowych.

Kolejną przydatną funkcjonlnością dla zwinnego projektu jest możliwość zatwierdzania etapów przez prowadzącego.
Zaakceptowanie powinno odbywać się po spotkaniu z prowadzącym, jednak taka funkcjonalność mogłaby dodatkowo pomóc w ocenie końcowej przebiegu projektu.

Wszystkie omówione modyfikacje da się wprowadzić do platformy nie tracąc przy tym jej dotychczasowych funkcjonalności.
W większości nie są to trudne do zaimplementowania funkcjonalności, jednak wymagają one przed wprowadzeniem dodatkowej, pełnejszej analizy.
Dodanie nowej roli i wprowadzenie dodatkowego udostępniania zasobów powinno być dobrze przemyślane i zaprojektowane przed implementacją.

\subsubsection{Podsumowanie}

Platforma może zostać rozszerzona, tak aby możliwe było z jej wykorzystaniem przeprowadzenie zwinnego projektu.
Modyfikacje nie powinny wpłynąć na obecne funkcjonalności platformy.
Dzięki takiemu podejściu narzędzie stało by się jeszcze bardziej uniwersalne.

Pomysł udostępnienia platformy dla zwinnych projektów nasunął się podczas badania gotowego narzędzia.
Jest to moment w którym platforma była gotowa i zweryfikowana w podstawowej wersji.
Dodanie na tym etapie pracy nowych funkcjonalności mogłby doprowadzić do destabilizacji platformy i nie przyiosło by pozytywnych rezultatów.
Stąd udostępnienie platformy do prowadzenia projektów w metodyce Agile jest proponowanym rozszerzeniem pracy magisterskiej.


\section{Podsumowanie}
\label{research_summary}

W przypadku symulacji zwinnego projektu, znany jest ogólny cel projektu, natomiast nieznany jest pełny jego zakres.
Nowe zadania definiowane są po zakończeniu poprzednich etapów.
W tym przypadku prowadzący pełni rolę Agile Product Owner’a, który doprecyzowuje treści zadań w miarę kolejnych zapytań studentów.
Taka symulacja ma za zadanie odzwierciedlić na grupie testowej rzeczywiste środowisko, z którym studenci spotkają się na co dzień w pracy zawodowej.




TODO: Coś więcej napisać o Agile, dodać matrix (eee czy na pewno?)

- być może nie powinni również móc usuwać etapów które zostały przedstawione i zatwierdzone w jakiś sposób przez prowadzącego??

- można rozszerzyć o projekty Agile i wtedy platforma staje sie super uniwersalnym narzędziem

- jak tworzy się przypadki testowe to na bieżąco trzeba tworzyć binarkę i sprawdzać (prowadzący musi robić swoją - tak najwygodniej)

- zaczyna się od testów i przemyślenia ich (platforma troche to narzuca od strony prowadzącego)

- najlepiej przemyslec calosciowa koncepcje

- potem krok po kroku dodawac etapy i testy i po dodaniu etapow i testow dodac binarke

- ale zawsze mozna dodac tez jakis test pozniej i przedefiniowac zadanie

- wychodzi takie pseudo TDD (przynajmniej dla prowadzacego, studenci moga dalej nie umiec pisac testow i zrobic i tak zle testy na koniec)

- wygodne narzedzie, wlasciwie testy byly napisane wiec grupa testowa skupila sie na rozwiazaniu problemu

- wielokrotnie odpalali binarki

- poprawiali bledy

- widac bylo regresje

- wazne jest zeby dobrze zdefiniowac przypadki testowe

- bardzo dobrze ze przypadki testowe sa udostepnione stuentom

- dzieki udostepnienieniu przypadkow moga oni w latwy sposob ocenic i rozumiec co jest nie tak

- najlepiej dodac wlasny komparator dla miedzy innymi niestabilnych algorytmów

- obecny komparator porownuje dwa pliki 1 do 1 i moze sluzyc do definicji wiekszosci zadan

- ale może okazać sie ze to zbyt mało

- po zbadaniu platformy doszly zmiany funkcjonalnosci (niewielkie) ulatwiajace prace

- okazalo sie ze platforma jest intuicyjna i pomocna

- testy na wysokim poziomie - latwe do zrozumienia

- testy nie blokuja implementacji co jest bardzo wazne

- nie ma znaczenia jak w srodku studenci napisza program, testy nie wymuszaja implementacji



