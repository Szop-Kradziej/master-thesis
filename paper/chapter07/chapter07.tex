\chapter{Badania}
\label{chapter:research}

Badania systemu służą do sprawdzenia wpływu zastosowania platformy na wykonanie projektu, w tym ocenę usprawnienia pracy studentów i wpływu na ich sposób uczestnictwa w projekcie.
Dodatkowo badania systemu pozwoliły na ocenę łatwości i intuicyjności obsługi interfejsów oraz ocenę zadowolenia użytkowników z działania platformy.
W celu zbadania platformy zasymulowano dwa projekty.

Oba zadania zostały przeprowadzone na grupie testowej, składającej się z absolwentów Politechniki Warszawskiej.
Opis motywów doboru członków grupy testowej znajduje się w podrozdziale \ref{research_group}.

Pierwsza z symulacji jest przeprowadzeniem projektu analogicznyego do historycznego zadania wykonywanego przez studentów w~semestrze zima 2018 w~ramach przedmiotu ”Podstawy Programowania”.
Pełna definicja problemu, przebieg symulacji oraz wnioski zostały przedstawione w podrozdziale \ref{research_penguins}.

Drugie zadanie sprowadza się do napisania programu umożliwiającego przekształcenie obrazu.
Ma ono na celu zbadanie użyteczności tworzenia procesów integracji programów w ramach platformy.
Opis tej symulacji znajduje się w podrozdziale \ref{research_image}.

Kolejny podrozdział zawiera wnioski z nadania absolwentom uprawnień administratora podczas testowania platformy.
Okazało się, że bardzo chętnie tworzyli oni własne etapy i testowali swój kod przy użyciu narzędzia.
Doprowadziło to do wniosku, że platforma mogłaby zostać rozszerzona w przyszłości (nie tracąc dotychczasowych funkcjonalności) tak, aby możliwe było również przeprowadzenie z jej użyciem projektów w metodyce Agile.

Ostatni podrozdział \ref{research_summary} zawiera podsumowanie przeprowadzonych badań.


\section{Opis grupy testowej}
\label{research_group}

W skład grupy testowej, dla której zostały przeprowadzone obie badane symulacje projektów wchodziło dwóch absolwentów Politechniki Warszawskiej.
Obie osoby ukończyły wydział Elektroniki i Technik Informacyjnych.

Absolwenci brali udział w projektach grupowych realizowanych na wydziale EiTI i znają podstawy programowania.
Dzięki temu mają pełne zaplecze wiedzy potrzebnej do realizacji każdego z symulowanych projektów.
Mogą oni dzięki swojemu doświadczeniu ocenić czy zastosowanie platformy byłoby użyteczne podczas prowadzenia projektów studenckich.
Absolwenci potrafią także wskazać różnice w prowadzeniu projektu bez użycia platformy oraz z jej wykorzystaniem.

Obie osoby posiadają również doświadczenie komercyjne, dzięki temu są w stanie ocenić narzędzie od strony technicznej i zasugerować dodatkowe funkcjonalności mające na celu zwiększenie użyteczności platformy.

Drugą rozpatrywaną grupą osób, która mogłaby wydawać się właściwą do przetestowania użyteczności platformy są obecni studenci.
Taki pomysł zespołu testowy został jednak odrzucony ze względu na ciągły proces udoskonalania narzędzia podczas badań.

W testowanej wersji MVP ciągle mogą wystąpić nieprzewidziane zachowania platformy.
Wynika to z faktu, że w początkowej fazie projektu ciężko jest wyeliminować wszystkie błędy.
Do ich całkowitego usunięcia trzeba utrzymywać i wspierać aplikację na bieżąco.
Takie niewielkie problemy pojawiły się podczas badania platformy przez grupę absolwentów i były na bieżąco naprawiane.

Udostępnienie studentom platformy bez przetestowania jej na mniejszej grupie mogłoby spowodować rezultat odwrotny do zamierzonego.
Zamiast wykorzystać zalety rozwiązania takie osoby mogłyby zniechęcić się ze względu na występujące błędy.
Sam czas naprawy błędów mógłby nie być satysfakcjonujący dla studentów i pogłębiać flustrację.
W przypadku studentów pierwszego roku przekazanie narzędzia na takim etapie pracy zdecydowanie utrudniłoby im wykonanie projektu.

Warto również zaznaczyć, że studenci zwłaszcza pierwszego roku nie mają szerokiego doświadczenia w uczestnictwie w projektach grupowych.
Stąd wprowadzenie dodatkowych narzędzi mogłoby wzbudzać u nich początkową niechęć.
Mogłoby okazać się, że dopiero po zebraniu większego doświadczenia mogliby właściwie ocenić wpływ platformy na proces ich pracy.

Samo doprowadzenie platformy do użytecznej formy i zaimplementowanie wszystkich niezbędnych funkcjonalności było czasochłonne.
A w przypadku studentów należałoby udostępnić im platformę na pełen semestr, co znacznie wydłużyłoby proces pisania pracy.

Przekazanie platformy w obecnej wersji dla studentów i monitorowanie ich interakcji z platformą podczas pełnego projektu jest jedną ze wskazanych możliwości kontynuacji pracy.
Dzięki analizie przebiegu rzeczywistego projektu uczelnianego można byłoby zrozumieć lepiej proces pracy studentów i uzupełnić platformę o dodatkowe, pożyteczne funkcjonalności.


\section{Symulacja projektu historycznego}
\label{research_penguins}

W przypadku symulacji zdefiniowanego zadania pełny cel projektu jest znany od samego początku.
Wykonywanie kolejnych etapów przybliża studentów do wyznaczonego początkowo celu.
Taki typ projektu jest bardzo często prowadzony w ramach różnych przedmiotów.
W ramach pracy został przygotowany i zdefiniowany na platformie projekt symulujący ten rodzaj zadania.


\subsection{Definicja projektu}

TODO


\subsection{Przebieg symulacji}

TODO


\subsection{Podsumowanie}

TODO


\section{Symulacja projektu z procesem integracji}
\label{research_image}


Przekształcanie obrazu


\subsection{Definicja projektu}

TODO


\subsection{Przebieg symulacji}

TODO


\subsection{Podsumowanie}

TODO


\section{Udostępnienie grupie testowej platformy z prawami administratora}

- do testow udostepniono platforme z prawami admina

- pozwalalo to na samodzielne tworzenie projektow

- mialo na celu lepsze wychwycenie nieprawidlowosci w dzialaniu platformy

- przetestowanie czy integracja FE z BE jest właściwa

- ocena UI i UX (intuicyjności platformy)

- WYNIKI:

- absolwenci chetnie tworzyli własne projekty

- z zaangażowaniem tworzyli przypadki testowe i programy spelniajace je

- WNIOSKI:

- platforma mogłaby posłużyć do monitorowania projektów w metodyce Agile

- polegałoby to na tym, że studenci na początku sem dostają luźno zdefiniowane zadanie bez szczegółów

- w trakcie trwania projektu doprecyzowują je z prowadzącym

- sami tworzą kolejne etapy i testy akceptacyjne

- prowadzący może w dowolnym momencie podejrzeć postępy prac

- projekty mogą być zupełnie różne i wydaje się że podejście indywidualne do każdego z zespołów jest nieuniknione

- rzeczywiście jest ale można je troche zautomatyzować i ograniczyć

- tak jak w przypadku tradycyjnego projektu tak i tu prowadzacy ma jednolity wglad w postepy studentów

- oczekuje sie ze etapy beda definiowane analogicznie w ramach kazdego z projektow

- na pierwszy rzut oka widac postepty prac poprzez stworzone etapy

- w zwizaku z tym w ramach pracy dokonano zgrubnej analizy przeprowadzenia tego typu projektu

- wymagałoby to jedank pewnych mniejszych lub większych modyfikacji platformy

- należałoby przeanalizować lepiej problemy stojące za przeprowadzeniem projektu w tej metodyce

- napewo:

    - dodanie nowego poziomu uprawnień

    - w modelu MVP zakłada się że prowadzący edytują tylko swoje projekty i nie zmieniają innych (na złość komuś)

    - do tej pory tylko student i prowadzący

    - należałoby dodać studenta z możliwością zarządzania wskazanym projektem

    - wymagało by to dodatkowej relacji w bazie "student w projekcie"

    - warto zauważyć, że taka rola dalej nie mogłaby mieć pewłnych uprawnień

    - trzeba byłoby przeanalizować i ustalić zakres uprawnień takiej roli

    - np studenci nie powinni miec możliwości usunięcia całego projektu

    - być może nie powinni również móc usuwać etapów które zostały przedstawione i zatwierdzone w jakiś sposób przez prowadzącego

    - lepsze zrównoleglenie

    - ponieważ  w tym momencie jedynym zasobem do którego strzeżony jest równoległy dostęp jest uruchamianie programów

    - lepsze statystki

    - na chwile obecną statystyki zbierane tylko z uruchomień programów

    - dla MVP zakłada się ze prowadzący definiuje projekt na początku semestru i udostępnia go studentom

    - zakład się ze prowadzący od poczatku ma pewne wymagania i wie do jakiej postaci chcialby aby studenci doprowadzili swoje aplikacje

    - może on dokonywać modyfikacji w trakcie trwania semestru, dodawać etapy i testy jednak takie modyfikacje nie powinny być duże

    - potrzeba było by zeby zbierać również informacje z tworzenia etapów oraz przechowywać informacje dla nieistniejących etapów

    - takie dane mogłyby chociaż nie koniecznie musiały posłużyć do oceny pracy studentów w takcie semestru

    - albo chociaż oceny zaangażowania w porównaniau do innych grup

    - dane na pewno mogą posłużyć do analizy procesu pracy studentów podczas takiego typu projektu i zrozumienia specyfiki takiej pracy

    - dodatkowe funkcjonalności

    - jak zostało powiedziane wcześniej zostawienie studentom do napisania testow akceptacyjnych moze skutkowac niskim pokryciem testowym aplikacji

    - prowadzacy mialby wglad do testow na wysokim poziomie

    - nie musialby wnikac bezposrednio i bez potrzeby  w kod studentow w celu zrozumienia co ma zrobic dany etap

    - samodzielne zrozumienie danych mimo wysokiego poziomu testow moglaby by jednak wciaz czasochlonne i klopotliwe

    - usprawnienieniem procesu mogloby byc daodanie funkcjonalności komentarza dla przypadków testowych

    - mozliwe ze warto byłoby dodać milestone - czyli zatwierdzenie etapów przez prowadzącego do danego momentu

    - zatwierdzanie dalej powinno obywać się po spotkaniu z prowadzącym

    - mogłoby to pomóc w ocenie końcowej kontaktu z prowadzącym podczas projektu

- w większości nie są to trudne modyfikacje

- wymagają jednak dodatkowej pełniejszej analizy

- jednak dodanie nawet drobnych funkcjonalności i przetestowanie ich jest czasochłonne

- dodatkowo dodanie nowej roli i ustalenie równoległego udostępniania większej ilość zasobów powinno być dobrze przemyślane i zaprojektowane przez implemetacją

- w tym trzeba ustalić dokładnie które endpointy wymagają dostępu i które zasoby powinny być lockowane

- te pomysly wyszly podczas badania gotowej platformy

- jest to moment w którym projekt MVP platformy jest gotowy i zweryfikowany

- dodanie na szybko (na tym etapie pracy) funkcjonalności doprowadziłoby do destabilizacji platformy i nie przyniosło pozytywnych rezultatów

- stąd udostępnienie platformy dla projektów w systematyce Agile jest proponowanym rozszerzeniem pracy magisterskiej



W przypadku symulacji złożonego środowiska, znany jest ogólny cel projektu, natomiast nieznany jest pełny jego zakres.
Nowe zadania definiowane są po zakończeniu poprzednich etapów.
W tym przypadku prowadzący pełni rolę Agile Product Owner’a, który doprecyzowuje treści zadań w miarę kolejnych zapytań studentów.
Taka symulacja ma za zadanie odzwierciedlić na grupie testowej rzeczywiste środowisko, z którym studenci spotkają się na co dzień w pracy zawodowej.


TODO: Coś więcej napisać o Agile, dodać matrix (eee czy na pewno?)


\section{Podsumowanie}
\label{research_summary}

- można rozszerzyć o projekty Agile i wtedy platforma staje sie super uniwersalnym narzędziem



