\chapter{Badania}
\label{chapter:research}

W celu zbadania platformy zasymulowano dwa projekty.
Oba zadania zostały przeprowadzone na grupie testowej, składającej się z absolwentów Politechniki Warszawskiej.
Pierwsze z nich jest projektem analogicznym do historycznego zadania wykonywanego przez studentów w~semestrze zima 2018 w~ramach przedmiotu ”Podstawy Programowania”.
Drugie zadanie sprowadza się do napisania programu umożliwiającego przekształcenie obrazu.

Badania systemu mają na celu sprawdzenie wpływu zastosowania platformy na wykonanie projektu, w tym ocenę usprawnienia pracy studentów i wpływu na ich sposób uczestnictwa w projekcie.
Badania odbyły się na co dwu osobowej grupie testowej, której zadaniem było wykonanie dwóch rodzajów projektów.
Dodatkowo badania systemu pozwoliły na ocenę łatwości i intuicyjności obsługi interfejsów oraz ocenę zadowolenia użytkowników z działania platformy.


\section{Opis grupy testowej}

Opis grupy testowej:
- absolwenci
- ukończyli EiTI
- mają doświadczenie w braniu udziału w projektach grupowych na wydziale (podczas studiów)
- znają podstawy programowania (nie trzeba ich uczyć)
- mają komercyjne doświadczenie
- mogą ocenić dzięki doświadczeniu czy rozwiązanie jest przydatne podczas studiów
- mogą powiedzieć jak platforma wpływa na ich pracę w porównaniu do projektu bez platformy

Dlaczego nie studenci:
- doprowadzenie platformy do postaci MVP jest czasochłonne
- MVP ciągle mogą wystąpić jakieś nieprzewidzanie zachowania błedy
- ciężko dać studentom nie przetestowaną na mniejszej grupie platformę
- źle działająca platforma mogłaby znstudenci pierwszego roku nie iechęcić studentów
- w przypadku studentów pierwszego roku mogłaby utrudnić znacznie im pracę zamiast pomóc
- nie da się wyeliminować wszystkich błędów na początkowej fazie projektu, trzeba utrzymywać i supportować aplikacje
- trzeba byłoby przy okazji nauczyć programowania osoby z pierwszego roku
- testy dla studentów pierwszego roku trwałyby cały semestr
- testowanie na studentach i dodanie dodatkowych funkcjonalności i usprawnień - rozszerzenie pracy
- studenci pierwszego roku nie mogą powiedzieć jak platforma wpływa na ich pracę w porównaniu do projektu bez platformy


\section{Symulacja projektu historycznego}

W przypadku symulacji zdefiniowanego zadania pełny cel projektu jest znany od samego początku.
Wykonywanie kolejnych etapów przybliża studentów do wyznaczonego początkowo celu.
Taki typ projektu jest bardzo często prowadzony w ramach różnych przedmiotów.
W ramach pracy został przygotowany i zdefiniowany na platformie projekt symulujący ten rodzaj zadania.


\subsection{Definicja projektu}

TODO


\subsection{Przebieg symulacji}

TODO


\subsection{Podsumowanie}

TODO


\section{Symulacja projektu z procesem integracji}

Przekształcanie obrazu


\subsection{Definicja projektu}

TODO


\subsection{Przebieg symulacji}

TODO


\subsection{Podsumowanie}

TODO


\section{Udostępnienie grupie testowej platformy z prawami administratora}


W przypadku symulacji złożonego środowiska, znany jest ogólny cel projektu, natomiast nieznany jest pełny jego zakres.
Nowe zadania definiowane są po zakończeniu poprzednich etapów.
W tym przypadku prowadzący pełni rolę Agile Product Owner’a, który doprecyzowuje treści zadań w miarę kolejnych zapytań studentów.
Taka symulacja ma za zadanie odzwierciedlić na grupie testowej rzeczywiste środowisko, z którym studenci spotkają się na co dzień w pracy zawodowej.


TODO: Coś więcej napisać o Agile, dodać matrix (eee czy na pewno?)


\section{Podsumowanie}



